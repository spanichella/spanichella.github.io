\documentclass[11pt]{article}
\pagestyle{plain}
\usepackage{xcolor}
\usepackage{lipsum}
\usepackage{setspace}
\renewcommand{\baselinestretch}{1.25} 
\usepackage{calc}
\usepackage{graphicx} 
\reversemarginpar
   \usepackage{graphicx}
\usepackage[paper=letterpaper, 
            marginparwidth=0.1in, 
            marginparsep=.1in, 
            margin=0.4in,
            left=0.4in,%
            right=0.4in,%
            includemp]{geometry}

\setlength{\parindent}{0in}

\usepackage[shortlabels]{enumitem}
% ============================================================
%:Markup macros for proof-reading
\usepackage{ifthen}
\usepackage[normalem]{ulem} % for \sout
\usepackage{xcolor}
\newcommand{\ra}{$\rightarrow$}
\newboolean{showedits}
\setboolean{showedits}{true} % toggle to show or hide edits
%\setboolean{showedits}{false} % toggle to show or hide edits
\ifthenelse{\boolean{showedits}}
{
	\newcommand{\meh}[1]{\textcolor{red}{\uwave{#1}}} % please rephrase
	\newcommand{\ins}[1]{\textcolor{blue}{\uline{#1}}} % please insert
	\newcommand{\del}[1]{\textcolor{red}{\sout{#1}}} % please delete
	\newcommand{\chg}[2]{\textcolor{red}{\sout{#1}}{\ra}\textcolor{blue}{\uline{#2}}} % please change
	\newcommand{\nbe}[3]{
		{\colorbox{#3}{\bfseries\sffamily\scriptsize\textcolor{white}{#1}}}
		{\textcolor{#3}{\sf\small$\blacktriangleright$\textit{#2}$\blacktriangleleft$}}}
}{
	\newcommand{\meh}[1]{#1} % please rephrase
	\newcommand{\ins}[1]{#1} % please insert
	\newcommand{\del}[1]{} % please delete
	\newcommand{\chg}[2]{#2}
	\newcommand{\nbe}[3]{}
}
%
\newcommand\rA[1]{\nbe{Reviewer A}{#1}{cyan}}
\newcommand\rB[1]{\nbe{Reviewer B}{#1}{olive}}
\newcommand\rC[1]{\nbe{Reviewer C}{#1}{magenta}}
\newcommand\ANS[1]{\nbe{Response}{#1}{teal}}
% ============================================================
%:Put edit comments in a really ugly standout display
%\usepackage{ifthen}
\usepackage{amssymb}
\newboolean{showcomments}
\setboolean{showcomments}{true}
%\setboolean{showcomments}{false}
\newcommand{\id}[1]{$-$Id: scgPaper.tex 32478 2010-04-29 09:11:32Z oscar $-$}
\newcommand{\yellowbox}[1]{\fcolorbox{gray}{yellow}{\bfseries\sffamily\scriptsize#1}}
\newcommand{\triangles}[1]{{\sf\small$\blacktriangleright$\textit{#1}$\blacktriangleleft$}}
\ifthenelse{\boolean{showcomments}}
%{\newcommand{\nb}[2]{{\yellowbox{#1}\triangles{#2}}}
{\newcommand{\nbc}[3]{
 {\colorbox{#3}{\bfseries\sffamily\scriptsize\textcolor{white}{#1}}}
 {\textcolor{#3}{\sf\small$\blacktriangleright$\textit{#2}$\blacktriangleleft$}}}
 \newcommand{\version}{\emph{\scriptsize\id}}}
{\newcommand{\nbc}[3]{}
 \newcommand{\version}{}}
\newcommand{\nb}[2]{\nbc{#1}{#2}{orange}}
\newcommand{\here}{\yellowbox{$\Rightarrow$ CONTINUE HERE $\Leftarrow$}}
\newcommand\rev[2]{\nb{TODO (rev #1)}{#2}} % reviewer comments
\newcommand\fix[1]{\nb{FIX}{#1}}
\newcommand\todo[1]{\nb{TO DO}{#1}}
\newcommand\on[1]{\nbc{ON}{#1}{red}} % add more author macros here
%\newcommand\XXX[1]{\nbc{XXX}{#1}{blue}}
%\newcommand\XXX[1]{\nbc{XXX}{#1}{brown}}
%\newcommand\XXX[1]{\nbc{XXX}{#1}{cyan}}
%\newcommand\XXX[1]{\nbc{XXX}{#1}{darkgray}}
%\newcommand\XXX[1]{\nbc{XXX}{#1}{gray}}
%\newcommand\XXX[1]{\nbc{XXX}{#1}{magenta}}
%\newcommand\XXX[1]{\nbc{XXX}{#1}{olive}}
%\newcommand\XXX[1]{\nbc{XXX}{#1}{orange}}
%\newcommand\XXX[1]{\nbc{XXX}{#1}{purple}}
%\newcommand\XXX[1]{\nbc{XXX}{#1}{red}}
%\newcommand\XXX[1]{\nbc{XXX}{#1}{teal}}
%\newcommand\XXX[1]{\nbc{XXX}{#1}{violet}}
% ============================================================


\makeatletter
\newlength{\bibhang}
\setlength{\bibhang}{1em}
\newlength{\bibsep}
 {\@listi \global\bibsep\itemsep \global\advance\bibsep by\parsep}
\newlist{bibsection}{itemize}{3}
\setlist[bibsection]{label=,leftmargin=\bibhang,%
        itemindent=-\bibhang,
        itemsep=\bibsep,parsep=\z@,partopsep=0pt,
        topsep=0pt}
\newlist{bibenum}{enumerate}{3}
\setlist[bibenum]{label=[\arabic*],resume,leftmargin={\bibhang+\widthof{[99]}},%
        itemindent=-\bibhang,
        itemsep=\bibsep,parsep=\z@,partopsep=0pt,
        topsep=0pt}
\let\oldendbibenum\endbibenum
\def\endbibenum{\oldendbibenum\vspace{-.6\baselineskip}}
\let\oldendbibsection\endbibsection
\def\endbibsection{\oldendbibsection\vspace{-.6\baselineskip}}
\makeatother


\usepackage{fancyhdr,lastpage}
\pagestyle{fancy}
%\pagestyle{empty}      % Uncomment this to get rid of page numbers
\fancyhf{}\renewcommand{\headrulewidth}{0pt}
\fancyfootoffset{\marginparsep+\marginparwidth}
\newlength{\footpageshift}
\setlength{\footpageshift}
          {0.2\textwidth+0.2\marginparsep+0.2\marginparwidth-2in}
%\lfoot{\hspace{\footpageshift}%
%       \parbox{2in}{\, \hfill %
%                    \arabic{page} of \protect\pageref*{LastPage} % +LP
%%                    \arabic{page}                               % -LP
%                    \hfill \,}}

% Finally, give us PDF bookmarks
\usepackage{color,hyperref}
\definecolor{darkblue}{rgb}{0.0,0.0,0.3}
\hypersetup{colorlinks,breaklinks,
            linkcolor=darkblue,urlcolor=darkblue,
            anchorcolor=darkblue,citecolor=darkblue}


\newcommand{\makeheading}[2][]%
        {\hspace*{-\marginparsep minus \marginparwidth}%
         \begin{minipage}[t]{\textwidth+\marginparwidth+\marginparsep}%
             {\large \bfseries #2 \hfill #1}\\[-0.15\baselineskip]%
                 \rule{\columnwidth}{1pt}%
         \end{minipage}}

\renewcommand{\section}[1]{\pagebreak[3]%
    \vspace{1.3\baselineskip}%
    \phantomsection\addcontentsline{toc}{section}{#1}%
    \noindent\llap{\scshape\smash{\parbox[t]{\marginparwidth}{\hyphenpenalty=10000\raggedright #1}}}%
    \vspace{-\baselineskip}\par}

\newcommand*\fixendlist[1]{%
    \expandafter\let\csname preFixEndListend#1\expandafter\endcsname\csname end#1\endcsname
    \expandafter\def\csname end#1\endcsname{\csname preFixEndListend#1\endcsname\vspace{-0.6\baselineskip}}}

\let\originalItem\item
\newcommand*\fixouterlist[1]{%
    \expandafter\let\csname preFixOuterList#1\expandafter\endcsname\csname #1\endcsname
    \expandafter\def\csname #1\endcsname{\csname preFixOuterList#1\endcsname\let\oldItem\item\def\item{\pagebreak[2]\oldItem}}
    \expandafter\let\csname preFixOuterListend#1\expandafter\endcsname\csname end#1\endcsname
    \expandafter\def\csname end#1\endcsname{\let\item\oldItem\csname preFixOuterListend#1\endcsname}}
\newcommand*\fixinnerlist[1]{%
    \expandafter\let\csname preFixInnerList#1\expandafter\endcsname\csname #1\endcsname
    \expandafter\def\csname #1\endcsname{\let\oldItem\item\let\item\originalItem\csname preFixInnerList#1\endcsname}
    \expandafter\let\csname preFixInnerListend#1\expandafter\endcsname\csname end#1\endcsname
    \expandafter\def\csname end#1\endcsname{\csname preFixInnerListend#1\endcsname\let\item\oldItem}}
 
\newlist{outerlist}{itemize}{3}
    \setlist[outerlist]{label=\enskip\textbullet,leftmargin=*}
    \fixendlist{outerlist}
    \fixouterlist{outerlist}

\newlist{lonelist}{itemize}{3}
    \setlist[lonelist]{label=\enskip\textbullet,leftmargin=*,partopsep=0pt,topsep=0pt}
    \fixendlist{lonelist}
    \fixouterlist{lonelist}

\newlist{innerlist}{itemize}{3}
    \setlist[innerlist]{label=\enskip\textbullet,leftmargin=*,parsep=0pt,itemsep=0pt,topsep=0pt,partopsep=0pt}
    \fixinnerlist{innerlist}

\newlist{loneinnerlist}{itemize}{3}
    \setlist[loneinnerlist]{label=\enskip\textbullet,leftmargin=*,parsep=0pt,itemsep=0pt,topsep=0pt,partopsep=0pt}
    \fixendlist{loneinnerlist}
    \fixinnerlist{loneinnerlist}

\newcommand{\blankline}{\quad\pagebreak[3]}
\newcommand{\halfblankline}{\quad\vspace{-0.5\baselineskip}\pagebreak[3]}

\newcommand\doilink[1]{\href{http://dx.doi.org/#1}{#1}}
\newcommand\doi[1]{doi:\doilink{#1}}

\providecommand*\url[1]{\href{#1}{#1}}
\renewcommand*\url[1]{\href{#1}{\texttt{#1}}}
\providecommand*\email[1]{\href{mailto:#1}{#1}}
\providecommand\BibTeX{{B\kern-.05em{\sc i\kern-.025em b}\kern-.08em
    \TeX}}
\providecommand\Matlab{\textsc{Matlab}}
\hyphenation{bio-mim-ic-ry bio-in-spi-ra-tion re-us-a-ble pro-vid-er}

\pagenumbering{roman}

\begin{document}
\pagenumbering{roman}
%\includegraphics[width=3.3cm, height=3.5cm]{images/s_panichella.jpg}\\
\vspace{-2mm}
\textsc{\fontsize{13}{12}\selectfont Sebastiano Panichella - Curriculum vitae \& major scientific achievements}\\
\vspace{-2mm}

%\newlength{\rcollength}\setlength{\rcollength}{1in}%
%\newlength{\spacewidth}\setlength{\spacewidth}{20pt}
%\newcommand\spacechar{$|$}

\noindent\begin{minipage}{0.2\textwidth}% adapt widths of minipages to your needs
\includegraphics[width=3.11cm, height=3cm]{images/PANC.jpg}
\end{minipage}%
\hfill%
\begin{minipage}{0.9\textwidth}\raggedright
\textsc{Contact Information}\\
{\small
%\textit{Address:} Department of Computer Engineering\\ \href{http://www.ing.unisannio.it/}{University of Sannio}\\
%RCOST- Palazzo ex Poste, Via Traiano\\
%82100 Benevento (Italy).\\\\
%\textit{Mobile:} +39 3881969673\\
%\textit{Tel.: +39 0824 305539} \\
%\textit{E-mail:} \email{spanichella@gmail.com}\\
%\textit{Home Page:} \href{http://www.ing.unisannio.it/spanichella}{www.ing.unisannio.it/spanichella}
\textit{Address:} \href{}{University of Bern (UniBe)}\\ 
Sch�tzenmattstrasse 14, 3012 Bern, Switzerland.\\ 
%\textit{Tel.: +39 0824 305539} \\
%\textit{Tel.: +41 (0) 58 934 41 56} \\
\textit{E-mail:} \email{sebastiano.panichella@unibe.ch} (or alternatively \email{spanichella@gmail.com})\\
\textit{Home Page:} \href{https://spanichella.github.io/}{https://spanichella.github.io/}\\
\textit{Google Scholar Ref:}\\ \url{https://scholar.google.it/citations?user=HiNuBFgAAAAJ\&hl=en\&oi=ao}\\
\textit{Short CV:} \href{https://spanichella.github.io/img/CV.pdf}{https://spanichella.github.io/img/CV-short.pdf}\\
\textit{Detailed CV:} \href{https://spanichella.github.io/img/CV.pdf}{https://spanichella.github.io/img/CV.pdf}}\\
\end{minipage}


\vspace{2.5mm}
\textsc{Education}
\vspace{1.5mm}

Sebastiano Panichella was born in Italy.
He received the PhD in Computer Science from the University of Sannio 
 defending the thesis entitled  \href{http://dx.doi.org/10.1109/ICSM.2015.7332519}{\textit{``Supporting Newcomers in Open Source Software Development Projects"}} (\textbf{July 18th 2014}). \textbf{Supervisors}: Prof. Massimiliano  Di Penta and Prof. Gerardo Canfora.

\vspace{2.5mm}
\textsc{Employment history \& Institutional responsibilities}
\vspace{1.5mm}
%During the PhD his work was supervised by 
%Prof. Gerardo Canfora and 
%Prof. Massimiliano Di Penta and Prof. Gerardo Canfora.

%\textbf{Major scientific achievements}
%\vspace{1mm}

Currently, he is a Senior Computer Science Researcher at the University of Bern (from \textit{\textbf{09-2024}}). Previously he was a Senior Computer Science Researcher and Lecturer in Software Engineering at ZHAW (from 08-2018 to 08-2024), a postdoc at the University of Zurich (\textbf{2014-11-01 - 2018-08-19}) in the lab of Prof Gall and part-time (External) Lecturer at the University of Zurich (between \textit{\textbf{2018-2022}}). \\
%\medskip\\ 
His  \textbf{research interests} are in the domain of Software Engineering (SE), cloud computing (CC), and Data Science (DS): DevOps (e.g., Continuous Delivery, Continuous Integration), Machine learning applied to SE, Software maintenance and evolution (with particular focus on Cloud, mobile, AI-based, and Cyber-physical applications). 
He authored or co-authored \textbf{around one hundred} (considering also demonstration, dataset 
  and poster) papers appeared in International Conferences and Journals (26 of them published during the postdoctoral experience at the University of Zurich).
 

\vspace{2.5mm}


\textsc{\fontsize{14}{12}\selectfont Major scientific achievements}


\textbf{Achievement 1 - Funding, Leadership of Projects, and Related Impact}. In the last decade, He has demonstrated his ability to receive funding and lead projects as a PI
and co-PI, on topics related to Data Science (e.g., Mobile Computing), Mining Software
Repositories, DevOps and MLOps, Human-computer Interactions, Software Evolution, and
Software Evolution and Testing for Cyber-physical and AI Systems:

\vspace{-2.5mm}
\begin{itemize}
  \item PI (at UniBe) of the Horizon EU  "Marie Sklodowska-Curie Actions-funded Doctoral Networks" for the project "InnoGuard: Hybrid and Generative Intelligence for Trustworthy Autonomous Cyber-Physical Systems".
\textbf{Total project 607,132.8 EUR (595,779.42 CHF)}. 
  \item PI of the H2020 EU project COSMOS: ``DevOps for Complex Cyber-physical Systems". \textbf{Total H2020 project 5MIL EUR, Sebastiano Panichella got direct funding for 770,000 EUR}. Web page: \href{https://www.cosmos-devops.org/}{https://www.cosmos-devops.org/}
  \item PI for a Hasler Foundation Project. The funding will support and complement the studies of a Ph.D. student working "Testing Unmanned Aerial Vehicles" in the context of the COSMOS H2020 project (contract no. 957254).
\textbf{Total project 50,000 CHF} 
  \vspace{-2mm}
   \item PI for the Doctoral funding at the SoE ZHAW. \textbf{Total project funding:\textbf{114,000 CHF}}.
    \vspace{-2mm}
  \item PI of the Innosuisse national project ``ARIES: Exploiting User Journeys and Testing Automation for Supporting Efficient Energy Service Platforms". \textbf{Total project funding: \textbf{500,000 CHF}}. Web page: \href{https://spanichella.github.io/projects/aries-devops/index.html}{https://spanichella.github.io/projects/aries-devops/index.html}
    \vspace{-2mm}
\item PI for
   %\chg{funded}{obtained funding for} 
   the project ``SwarmOps: Human-sensing based MLOps for Collaborative Cyber-physical systems" (2024-2028).  (\textbf{Total project 667,280 CHF}). \\Web page: \href{https://spanichella.github.io/projects.html}{https://spanichella.github.io/projects.html}
      \vspace{-2mm}
  \item Co-PI for
   the SURF-MobileAppsData SNF project. \textbf{Total SNSF (CHF) 349,926}. Web page: \href{http://www.ifi.uzh.ch/en/seal/people/panichella/SNF-Projects.html}{http://www.ifi.uzh.ch/en/seal/people/panichella/SNF-Projects.html}
\vspace{-2mm}
\end{itemize}

A key aspect that attract collaborations and funding are his accessible and replicable research results. Tangible examples are the frameworks widely
used by academia and industry, developed as follow-up development activities of approaches
proposed in top publications in A and A* conferences and top journals. To provide relevant
examples, his team has developed frameworks now used as a reference to test and monitor the safety
states of Unmanned aerial vehicles (UAVs) and self-driving cars (SDCs), which requires both
sensor-based and digital twins analysis of simulated and real autonomous systems. These
frameworks are at the core of new and original tool competitions organized in past and recent
international events, where he acted as general/workshop chair:
\begin{itemize}
	\item https://shonan.nii.ac.jp/seminars/204/
	\item https://conf.researchr.org/committee/icst-2025/icst-2025-organizing-committee
	\item https://conf.researchr.org/track/icst-2025/icst-2025-tool-competition--self-driving-cartesting
	\item https://conf.researchr.org/track/icst-2025/icst-2025-tool-competition--uav-testing
	\item https://sbft24.github.io/tools/ , https://sbft23.github.io/tools/ , https://sbst22.github.io/
	\item https://nlbse2024.github.io/index.html , https://nlbse2023.github.io/tools/ 
\end{itemize}
Some of these frameworks where the concrete results of his recently finished COSMOS EU
project. Interestingly, the European Commission's Innovation Radar Assessment of the project
stated that the innovations developed in the COSMOS H2020 project (https://www.cosmosdevops.
org/) have been analyzed by the European Commission's Innovation Radar
(https://innovation-radar.ec.europa.eu/) and categorized as "Market Maturity" and "Market
Ready" (https://innovation-radar.ec.europa.eu/methodology/\#maturity-info) as reported in the
selected EU portal (https://innovation-radar.ec.europa.eu/innovation/57508). Very few projects
have been marked with such evaluation. The main significance of such recognition results in a
concrete invitation to build a startup/company on topics related to the COSMOS project.

\textbf{Achievement 2 - Contributions to the Research Field, the Community, and Education}
\vspace{-2.5mm}
\begin{itemize} 
\item \textbf{Impact in the Field and the Community:} 
Over the last six years, his projects, research publications, and collaborations have positioned him as a leading researcher in mining software repositories, automated development, testing, and monitoring of cyber-physical and AI-enabled systems. His work complements research at other Swiss universities.  His research results are published at leading conferences where his research areas find
applications (e.g., ICSE, ASE, ESEC/FSE, ICST, etc.) and internationally renowned software and
systems engineering journals (e.g., TOSEM, TSE, EMSE, JSS, etc.). In terms of community services,
He regularly serves as the organizer, PC member, and reviewer for international conferences, workshops
, and scientific journals, and I am very well connected and/or organize regularly international events
(conferences and workshops\footnote{https://adevops4iot.github.io/
-	https://sbst22.github.io/ - https://sbft23.github.io/ - https://sbft24.github.io/ -	https://nlbse2022.github.io/ - https://nlbse2023.github.io/ - https://nlbse2024.github.io/}), involving numerous national and international leading scientists
and research institutions in Switzerland and abroad.
Complementary, his collaborations involve
a dense set of flagship companies such as automotive (e.g., AICAS), aviation (GMV), robotics (e.g.,
Anybotics), e-health (Siemens Healthcare), and other intelligent system domains.
Key Contributions:
\begin{itemize}
\item 

\textbf{Mining Software Repositories \& DevOps for Software Systems and CPSs}: he analyzed diverse software data (e.g., Apache projects, mobile app logs) to support development activities. His recent studies has been on AI-based systems and CPSs (e.g., in automotive, healthcare, robotics), which need better adaptability to dynamic environments. As the technical coordinator of the COSMOS H2020 project (link), He defined DevOps practices for testing and monitoring CPSs in critical sectors. Through studies, workshops, and special issues, He explored DevOps, MLOps, and Digital Twin strategies, especially for continuous delivery and test automation. Current projects aim to enhance CPSs monitoring, leveraging operational data (e.g., sensors, models) for security and safety.
\textbf{AI for Software Engineering \& Software Engineering for AI-CPSs}: He applied AI (e.g., ML, NLP) to predict bugs, automate tests, and address inconsistencies. In the SURF SNF project, He combined AI and human input analysis to enhance software maintenance and testing. His recent ARIES Innosuisse Project focused on AI-driven tools for efficient energy systems. Upcoming projects (SwarmOps, Innoguard) will advance the adaptability of AI-CPSs to human needs.
  \vspace{-2mm}
\end{itemize}

 \item \textbf{Top Selected publications that influenced (or will influence) the research field}:
\begin{enumerate}
\item 
S. Panichella, A. Di Sorbo, Emitza Guzman, A. Visaggio, G. Canfora and H. Gall: How Can I Improve My App? Classifying User Reviews for Software Maintenance and Evolution. International Conference on Software Maintenance and Evolution. 2015
\item G. Grano, C. Laaber, A. Panichella, and S. Panichella: Testing with Fewer Resources: An Adaptive Approach to Performance-Aware Test Case Generation. Transactions on Software Engineering. 2019
\item A. Di Sorbo, F. Zampetti, A. Visaggio, M. Di Penta, and S. Panichella: Automated Identification and Qualitative Characterization of Safety Concerns Reported in UAV Software Platforms. Transactions on Software Engineering and Methodology. 2022
\item Z., Fiorella; Tamburri, D. ; Panichella, S.; Panichella, A.; Canfora, G.; Di Penta, M.: Continuous Integration and Delivery practices for Cyber-Physical systems: An interview-based study.    Transactions on Software Engineering and Methodology. 2022
\item F. Zampetti, R. Kapur, M. Di Penta, S. Panichella: An Empirical Characterization of Software Bugs in Open-Source Cyber-Physical Systems.    Journal of Systems & Software. 2022
\item C. Birchler, S. Khatiri, B. Bosshard, A. Gambi, S. Panichella: "Machine Learning-based Test Selection for Simulation-based Testing of Self-driving Cars Software". Empirical Software Engineering. 2023.
\item S. Khatiri, S. Panichella, P. Tonella: Simulation-based Test Case Generation for Unmanned Aerial Vehicles in the Neighborhood of Real Flights. International Conference on Software Testing, Verification and Validation. 2023
\item C. Birchler, T. Kombarabettu Mohammed, P. Rani, T. Nechita, T. Kehrer, S. Panichella: How does Simulation-based Testing for Self-driving Cars match Human Perception? ACM International Conference on the Foundations of Software Engineering. 2024
\item S. Panichella contributions to the book: "Large Language Models in Cybersecurity and Cyberdefense: Novel Threats and Mitigations Perspectives". Chapters: "Vulnerabilities Introduced by LLMs through Code Suggestions",  "Enhancing Security Awareness and Education for Large Language Models". 2024 
\item C. Birchler, S. Khatiri, P. Rani, T. Kehrer, S. Panichella: A Roadmap for Simulation-Based Testing of Autonomous Cyber-Physical Systems: Challenges and Future Direction.    Special issue "A 2030 Roadmap for Software Engineering" in  Transactions on Software Engineering and Methodology. 2025
  \vspace{-2mm}
\end{enumerate}
  \vspace{-2mm}
  \item 
  \textbf{Contributions to Education and Supervision of Junior Researchers}: 
  He started teaching as a student assistant at the University of Salerno and later independently during his PhD at the University of Sannio. At UZH, he led the "Software Maintenance and Evolution" course and continued as an external lecturer after joining ZHAW. His teaching expanded at ZHAW with courses including Software Development, DevOps, Cloud Computing, and Java Programming. Recently at UniBe, He taught the Software Skills Lab (topics: Java, Linear Data Structures, Graphs, Trees, Sets, and Maps) and co-lecture the Software Engineering course (topics: DevOps and AI-based systems). He also offers research seminars and supervises student projects in DevOps and software engineering.
Over the years, He contributed to \textbf{educational chapters and am co-authoring a book on "DevOps for Cyber-Physical Systems" (targeted for 2025), involving 30+ international researchers}. He \textbf{supervised} 14 undergrad students, 22 MSc students and currently/recently \textbf{supervised (or co-supervised)} the work of 8 research assistants, and 9 PhD students (6 of them during the postdoctoral experience at the University of Zurich), with new projects in 2025 adding 5 PhD students. Many of these works resulted in peer-reviewed publications, including award-winning theses such as Pooja Rani's Best PhD Thesis at UniBe in 2023.

  
  \vspace{-2mm} 

\end{itemize}

\textbf{Achievement 3 - Awards, Tools \& Industrial and Academic Collaborations}
%\vspace{-2.5mm}

Ensuring replicability and impactful research results is central to \href{https://spanichella.github.io/collaborations.html}{fostering academic and industrial collaborations}. His approach leverages diverse strategies tailored to partners and research groups, including co-supervision, internships (e.g., Sajad Khatiri's PhD internship at Anybotics), participation in national (Inno Suisse/SNF) and EU (Horizon) projects, and course co-design. Additionally, He organizes conferences and workshops, such as serving as general chair for the 2025 International Conference on Software Testing, Verification, and Validation. 
Over the past decade, He built extensive collaborations with industrial and research organizations across various domains, including Genedata (biology computation), Stadler (rail), LEDCity (energy efficiency), Siemens AG and Healthcare (DevOps for ICT/healthcare), and more. His team ensures accessible and replicable research outputs, such as frameworks adopted in academia and industry, derived from top-tier publications in A/A* conferences and journals. 
 Dr. Sebastiano Panichella published over 100 published papers in international conferences and journals, many of which have earned best paper awards or nominations. His work has fostered impactful collaborations with academic and industrial partners globally, focusing on innovative solutions in software engineering and DevOps. 
 
 \textbf{Key Achievements and Recognition}:
\begin{itemize}
 \item \textbf{Awards - Complete list at \href{https://spanichella.github.io/awards.html}{https://spanichella.github.io/awards.html
}}: He received 4 tools awards (and nominations) as well as 12 best paper awards and best paper nominations\footnote{\url{https://spanichella.github.io/awards.html}}.
 %   \vspace{-2mm}
  \item The paper [S. Panichella, A. Di Sorbo, E. Guzman, C. Visaggio, G. Canfora, H. Gall: How can I improve my app? Classifying user reviews for software maintenance and evolution. ICSME 2015], which originated the idea behind his SNF project, is one of the \textbf{most cited papers of ICMSE 2015} (as reported in Google Scholar), with over \textbf{500 citations} in 9 years.   
 %\vspace{-2mm}
  \item The paper ICPC wrote during the bachelor studies of Dr. Panichella-[G. Capobianco, A. De Lucia, R. Oliveto, A. Panichella, S. Panichella: On the role of the nouns in IR-based traceability recovery. ICPC 2009: 148-157] is \textbf{among the most influential papers of ICPC in the last decade [period 2009-2019]}.
\item 
    \textbf{Top Researcher Rankings}: He ranked among the top 20 most impactful software engineering researchers worldwide (2019, 2021) by the Journal of Systems and Software. Included in Stanford University's top 2\% scientists in his field (2022, 2023) and recognized in the top 0.5\% by ScholarGPS (2024).
\end{itemize}

 \textbf{Collaborative Research Highlights}:
\begin{itemize}
\item 
\textbf{Collaboration with industrial organizations:} Over the years, I have established collaboration with researchers from Switzerland (UZH, USI, ZHAW, ETH, etc.) and abroad involving both academic and industrial organizations in USA (e.g., Washington State University), Canada (e.g., Polytechnique Montreal), Japan (e.g., Sony), etc.: \url{https://spanichella.github.io/collaborations.html}
\item   \textbf{Frameworks \& Global Initiatives}:  He developed \href{https://spanichella.github.io/tools.html}{DevOps testing frameworks} and \href{https://spanichella.github.io/publications.html}{published} taxonomies on safety hazards and accidents in UAVs with collaborators from the University of Zurich (UZH) and Universit� della Svizzera italiana (USI).
 He worked on testing and monitoring techniques for self-driving cars in collaboration with institutions such as Delft University of Technology and IMC University of Applied Sciences Krems. The research includes using Virtual Reality (VR) to study human perception of cyber-physical systems (CPS). He organized the Shonan Meeting (2023) on DevOps for CPSs, involving over 50 experts. He plans to publish a book on best practices and future directions in CPS development. \textbf{Future Directions}. Dr. Panichella aims to expand collaborations within Switzerland and internationally, advancing research on DevOps, AI, and CPS to tackle emerging challenges in autonomous systems and digital twins.
\end{itemize} 

 
\medskip \medskip
\textsc{Supervision (or Co-Supervision) of researchers at graduate \& postgraduate levels}
\medskip \\
%\vspace{-5mm}
He \textbf{supervised} 14 theses of undergrad students, theses (or projects) of 22 MSc students and has \textbf{supervised (or co-supervised)} the work of 8 research assistants, and 10 PhD students (6 of them during the postdoctoral experience at the University of Zurich), which published in relevant conference and journal venues. A complete list on advised researchers and papers accepted can be found at
\href{https://spanichella.github.io/teaching\_and\_advising.html}{https://spanichella.github.io/teaching\_and\_advising.html}\\ %\medskip 

\textsc{Teaching (ZHAW, UZH, and UniBe) activities:}
\medskip \\
\textbf{University of Bern:}
\medskip \\
- Software Skills Lab - Topic "Java Crash Course, Linear Data Structures, Graphs and Trees, Sets and Maps, (Data Structure) Algorithms" - 2024. \\
- Software Engineering Course - Topic "DevOps and testing AI-based cyber-physical systems" - 2022, 2023, 2024.\\  
\medskip \\ 
\textbf{University of Zurich:} 
\medskip \\
-   Lecturer and co-lecturer for the Software Maintenance and Evolution course in 2014, 2015, 2016, 2017, 2018, 2019, 2020, 2021, 2022.   \textit{Learning Goals}: During the course Sebastiano teach to
the students the foundations of software evolution and maintenance. This includes successful but aged software systems (i.e. legacy software), object-oriented reengineering, refactoring, change patterns, empirical analysis of software, classification/prediction models, software quality analysis. This course also discusses analysis platforms and tools, test case generation and continuous delivery technologies in the context of autonomous systems (e.g., drones and self-driving cars).
\medskip \\
\textbf{Zurich University of Applied Sciences (ZHAW):}
\medskip \\
- Software development 2 / project module 4 FS24 (In german ``Software Entwicklung 2 / Projektmodul 4") - 2024, 2025.\\
- DevOps Testing for Complex Systems - 2023, 2024. \\
- Cloud Computing course - CCP2 2020\\
- INF-Prog1 2020. Learning Goals: The main features of the Python program language.\\
- Co-lecturer for the CAS Information Engineering in 2018, 2019, 2020.
Learning Goals: Python program language.\\
- Lab Instructor for the Programming course in Java in 2018, 2019, 2020.
Learning Goals: Java program language.
\medskip 
\vspace{-5mm} 

\href{http://www.unisa.it}{\textbf{University of Sannio}}:
\medskip\\
- \textit{Lab Instructor} (December 2013) for the Programming Techniques course of Professor Gerardo Canfora\\   \textit{Learning Goals}:   The Languages ​​and Grammars, JavaCC parser.\\
- \textit{Teaching Assistant } for the Software Engineering course of Prof. Massimiliano Di Penta:\\   \textit{Learning Goals}:   
Recovering Traceability Links via Information Retrieval Methods
\medskip\medskip \\
\textsc{Memberships in panels, boards, and individual scientific reviewing activities}

\medskip 

\textbf{Reviewer/opponent of Ph.D. Dissertations:}
\begin{innerlist}
   \item External examinator of Ph.D. Dissertation   by Adriano Torres at at University of Adelaide (2024).
   \item External examinator of Ph.D. Dissertation   by Zainab Javed at at National University of Computer and Emerging Sciences, Islamabad, Pakistan (2024).
   \item External Review of Ph.D.research proposal  by Mme Zid at at Polytechnique Montreal, Institute of Computer Science (August 2022).
   \item Reviewer/opponent of a Ph.D. Dissertation   of Nitish Shriniwas at University of Bern, Institute of Computer Science (March 2022).
   \item Reviewer/opponent of a Ph.D. Dissertation  at University of Tartu, Institute of Computer Science (2019/2020)
\end{innerlist}
\medskip 


\textbf{Editorial Board Member of International Journals:}
\begin{innerlist}
   \item \emph{
              \href{http://onlinelibrary.wiley.com/journal/10.1002/(ISSN)2047-7481��}
                   {Journal of Software: evolution and process}}.
   \item \emph{Transactions on Software Engineering and Methodology}.
\end{innerlist}
\medskip 
\textbf{Editor of special Issues at International Journals:}
\begin{innerlist}
\item Editor of Software Track special Issue at Journal of Science of Computer Programming on ``SBFT'23: Search-Based and Fuzz Testing Tools". 2023
\item Editor of Software Track special Issue at Journal of Science of Computer Programming on SBST’22: Search-Based Software Engineering – Tools. 2022
\item Editor of special issue at Science of Computer Programming Journal on NLP-based software engineering, 2022.
\item Editor of the 'Software Engineering for Mobile Applications' special Issue at EMSE Journal, 2018-07.
\item Editor of the 'User Feedback and Software Quality in the Mobile Domain' special Issue at IST Journal,  2018-06.
	
\end{innerlist}
\medskip 
%\on{Get rid of all the emphasis (no italics needed).}\\
\textbf{Organising committee member of International Conferences and Workshops:}
\begin{innerlist}
 \item \href{https://spanichella.github.io/services.html}{\textit{Program Committee member}} of ICSE, FSE, ASE, ICSME, ICST, ICSOFT, SSBSE, ICPC, SSBSE, SBST, SANER, MSR, WAISE, MaLTeSQuE, SEAA, SATToSE, VST, RoSE, QUATIC. \\\textbf{Complete list at}: \href{https://spanichella.github.io/services.html}{https://spanichella.github.io/services.html}
 \end{innerlist}

%\textbf{Session Chair of International Conferences:}
%\begin{innerlist}

 %      \item \emph{\href{http://saner.aau.at/}
  %                 {\textit{of the 24th IEEE International Conference on Software Analysis, Evolution, and Reengineering (SANER 2017)}}, Austria.}

%\end{innerlist}

%\textbf{Web Chair}
%\begin{innerlist}
  % \item \emph{
             % \href{http://icpc2014.usask.ca/‎}
                   %{21st International Conference on Program Comprehension %(ICPC 2013)}}, San Francisco, California, USA.\\
%\end{innerlist}

\textbf{Reviewer for the following International Journals:}
\begin{innerlist}
\item \emph{Scientific Reports, Nature - Empirical Software Engineering - Transactions on Software Engineering - Transactions on Software Engineering and Methodology - Journal of Systems and Software - Information and Software Technology - Journal of Software: Evolution and Process - Science of Computer Programming - Journal of Computer Science and Technology - Transactions on Mobile Computing - Communications of the ACM - Software Testing, Verification and Reliability - Journal of Object Technology - Transactions on Services Computing - Software: Practice and Experience journal
- Communications of the ACM}. \\ \textbf{Complete list at}: \href{https://spanichella.github.io/services.html}{https://spanichella.github.io/services.html}
\end{innerlist} 

\medskip 
\textbf{External Reviewer of Grant Applications}
\begin{innerlist}
   \item \emph{External Reviewer of \href{https://anid.cl/about-us/}{Regular Fondecyt National Projects}. Research Projects Sub-Directorate (SPI) of the National Agency for Research and Development (ANID) of the Ministry of Science, Technology, Knowledge and Innovation of Chile}
   \item \emph{External Reviewer of \href{https://prin.mur.gov.it/}
                   {PRIN (National Project)} and member of the CNVR (National Committee for the Evaluation of Research) for the Ministry for University and Research (MUR) in Italy, aimed at financing public research projects. }

   \item \emph{
              \href{http://www.frqnt.gouv.qc.ca/accueil}
                   {External Reviewer of projects submitted in the Quebec-Flanders bilateral research cooperation program}}
\item \emph{
              \href{}
                   {External Reviewer of projects submitted in the Mitacs Accelerate research program\\\\}}
                   
\end{innerlist}


\textsc{Active memberships in scientific societies, fellowships in renowned academies}

%\textbf{Member of associations:}
\begin{innerlist}
\item Nominated as Management Committee Member to represent the \textbf{COST Action CA22137 in Switzerland. Member of the COST (European Cooperation in Science and Technology)}: WG4 on "Optimisation under uncertainty" - https://www.cost.eu/actions/CA22137/
\item Member of the \textbf{EU Sparc Robotics group} - https://sparc-robotics-portal.eu
\item Member of the \textbf{ZHAW Digital Futures Lab} - https://www.zhaw.ch/en/focus-topics/zhaw-digital/digital-futures-lab/ 2023-2024
\item  He is a \textbf{member of IEEE/ACM}.
\end{innerlist}

\medskip\medskip 
\textsc{General Chair of International Conferences:}
\medskip 
\begin{innerlist}
 \item \href{https://conf.researchr.org/home/icst-2025}{\textit{International Conference on Software Testing, Verification and Validation - ICST 2025}} 
 \end{innerlist}
\medskip\medskip 
\textsc{Organising committee member of International Conferences and Workshops:}
\medskip 
\begin{innerlist}
 \item \href{https://spanichella.github.io/services.html}{\textit{Program Committee member}} of ICSE, FSE, ASE, ICSME, ICST, ICSOFT, SSBSE, ICPC, SSBSE, NLBSE, SBFT, SBST, SANER, MSR, WAISE, MaLTeSQuE, SEAA, SATToSE, VST, RoSE, QUATIC, etc.  \\\textbf{Complete list at}: \href{https://spanichella.github.io/services.html}{https://spanichella.github.io/services.html}
 \end{innerlist}
\medskip 
\textbf{Organising research workshops:}
\begin{innerlist}
\item Chair of the \textit{International Workshop on Artificial Intelligence in Software Testing (AIST)} - Collocated with ICST 2024
\item Chair of the \textit{Workshop on Natural Language-Based Software Engineering Workshop} (NLBSE) - Collocated with ICSE 2022, 2023, 2024, 2025
\item Chair of  the \textit{Workshop on Search-Based Software Testing} (SBST) - Collocated with ICSE 2022, ICSE 2023
\item Chair of the \textit{Workshop on DevOps Testing for Cyber-Physical Systems - Collocated with ICST 2021 \\(https://devops4cps-testing.github.io/)} 
\item \href{}
{Chair of the Tool Competition at SBST and SBFT (2020, 2021, 2024)} 
       \item \emph{\href{http://cnax.servicelaboratory.ch/}
                   {\textit{Chair of the first International Workshop on Cloud-Native Applications Design and Experience - CNAX 2018
Co-located with UCC 2018 and BDCAT 2018 conferences}}, Zurich, Switzerland.}
\end{innerlist}
\medskip 
\textbf{Keynote Speaker of International Conferences and co-located events:}
\begin{innerlist}
\item Speaker at the AIST workshop, co-located with the International Conference on Software Testing, Verification and Validation - 2023  \\(https://aistworkshop.github.io/\#keynote) 
\item Speaker at the \href{https://safecomp2021.hosted.york.ac.uk/wp-content/uploads/2021/08/DepDevOps_2021_programme.pdf}{Workshop on Dependable DevOps} co-located with the SafeComp conference, 2021.
\item Speaker at the Workshop on Validation, Analysis and Evolution of Software Tests - VST 2018  \\(http://vst2018.scch.at/\#program) 
\end{innerlist}
\medskip 
\textbf{Research Meetings}
\begin{innerlist}
   \item 

   \emph{
              \href{http://www.nii.ac.jp/}
                   {Sebastiano Panichella Led a a Shonnan meeting with the \href{http://www.nii.ac.jp/}{National Institute of Informatics} (NII), Japan, on the topic:  \href{https://shonan.nii.ac.jp/seminars/204/}{"DevOps for Cyber-physical Systems"}.
}}\\
%\item  \emph{Sebastiano Panichella was invited by the \href{http://www.adesso.de/de/}{Adesso company}, Switzerland, to participate in \textit{``Adesso Quartalsmeeting" 2016} (Zurich).}
\end{innerlist}
%\vspace{8.5mm}
\medskip\medskip\medskip
%\vspace{-2.5mm}
%\textsc{Awards - Complete list at \href{https://spanichella.github.io/awards.html}{https://spanichella.github.io/awards.html}}
%\medskip

%\textbf{Award as Reviewer}: Distinguished Reviewer Award SATToSE 2017, at SANER 2018, MSR 2022, TSE 2023.


%\footnote{In papers marked with (*)  the authors are listed in alphabetic order}: %When such a rule is not followed, authors are listed by contribution.
%\textbf{Best paper award:} ICPC 2011, MaLTeSQuE 2018.\\ 
%\textbf{Best tool award}: ICPC 2014, SANER 2018. 

%\vspace{1.5mm}

%\textbf{Nominations for Best Paper}: ICSME 2020, ICSSP 2020, (2) SANER 2018, SANER 2017, ICSME 2014, ICPC 2014, ICSM 2013, ICST 2013.

\medskip 

%\newpage

%\textbf{RESEARCH PARTNERS \& COLLABORATIONS established in the last years}:\\
%(\textcolor{blue}{collaborations summarized at \href{https://spanichella.github.io/collaborations.html}{https://spanichella.github.io/collaborations.html})}
%
%\textbf{Collaboration with researchers in the supervision or co-supervision of PhD students and research assistants}:
% Currently, he is advising at the Zurich University of Applied Science (ZHAW), the PhD  work of Sajad Khatiri (with Paolo Tonella from University of Lugano) and other 3 research assistants, on research concerning DevOps for complex systems (funded in EU and national projects).
% With Prof. Oscar Nierstrasz he is also co-advising the research work of Pooja Rani, PhD student at University of Bern, Switzerland (from 2018), on topics concerning code comments analysis and assessment. In the past years, Dr. Panichella advised the work of a research assistant (Diego Martin) at the ZHAW, co-supervised with Prof. Gall the research work of 5 PhD students at the University of Zurich, 1 PhD student at the University of Sannio (Italy) with Prof. Gerardo Canfora, 1 at University of Chinese Academy of Sciences, Beijing, China (Laboratory for Internet Software Technologies).
% More information on his advising or co-advising experience can be found at \href{https://spanichella.github.io/\#teaching}{https://spanichella.github.io/\#teaching}
%
%\textbf{Established Research partners interested to collaborate to projects and proposals with Dr. Panichella}: Timo Kehrer (University of Bern); Davide Scaramuzza (University of Zurich); Stefano Mintchev (ETH); Paolo Tonella (University of Lugano)
% University of Luxembourg, Dr. Bianculli and Dr. Pastore (Testing, Requirement engineering, and formal verification); Simula, Department of Engineering, Dr. Shaukat (Cyber-Physical Systems of Systems);  Delft University of Technology,
%Dr. Zaidman (automated testing); Tampere University of Technology, Prof. Davide Taibi (DevOps, cloud computing); Prof.
%Robles (Mining Software Repositories techniques). With most of the aforementioned (excluding Prof. Robles, that recently published
%with us a paper at Onward 2018) partners we collaborate on strategic EU and national projects. 

%\textbf{Collaborations with the industrial partners (via the involvement in research papers, projects and proposals)}:\\
%- Siemens AG and Siemens Healthcare GmbH	Germany	2019-05-Today\\
%- BOND	Switzerland	2019-10-Today\\
%%- Helio	Switzerland	2019-10-Today\\
%- Intelligentia S.r.l.	Italy	2020-01-Today\\
%- AICAS GmbH	Germany	2020-01-Today\\
%- Q-media s.r.o.	Czech Republic	2020-01-Today\\
%- Unparallel Innovation LDA	Portugal	2019-01-Today\\
%- The Open Group (Scott Hansen) Belgium 2019-01-Today\\
%- GMV https://www.gmv.com Spain 2019-01-Today\\
%- https://www.intelligentia.eu (Italy); 2019-01-Today\\
%- DENSO https://www.denso.com/de/en/innovation/ 2019-01-Today\\
%- Haidar Osman (Senior Data Scientist - Swisscom, Switzerland); 2018-Today
%- Red Hat Switzerland 2018-Today\\
%- https://vshn.ch/en/ Switzerland 2018-Today\\
%- https://ikubinfo.al/ Austria 2018-Today\\
%- Daniele Romano (ING Netherland); 2017-Today\\
%- Junji Shimagaki (Sony Mobile Communications); 2016-2018\\
%- etc.

%\textbf{Established collaborations with Robotics labs \& Exploitation Plans:}\\
% Dr. Panichella is currently collaborating with Prof. Scaramuzza on research concerning robotics and software engineering, primary on the co-supervision of thesis in the intersection of software engineering and robotics systems development. He is also fostering collaborations with robotics experts at ZHAW, focused on drones development.
% Dr. Panichella plans to exploit tools and technologies developed in his research projects, involving industrial organizations in Switzerland (e.g., Red Hat), including companies
%in which CPS and AI technologies are critical for their business. 
%Dr. Panichella plan to disseminate his projects results within software engineering, robotics courses at the Bachelor/Master level, thus fostering
%technological transfer to future employees of enterprises in the territory. Finally, Dr. Panichella is committed to establish the open source tools developed within his projects, to promote industrial and open source ecosystems and collaborations.\\


%\textbf{DATASET \& TOOLS:}\\
%A comprehensive and updated list of shared datasets/tools to the research community by Sebastiano is available at\\ https://spanichella.github.io/tools.html\\

%\vspace{3.5mm}

%\textbf{Internships}
%\begin{innerlist}
%   \item \emph{
%              \href{http://www.polymtl.ca/‎}
%                   {May-July 2013 was visiting researcher at the Ecole Polytechnique de Montr\`{e}al, Canada. Supervisor: Prof. Giuliano Antoniol}}
%\end{innerlist}
%\vspace{5.5mm}

%\textsc{Spoken Languages}\\

%Sebastiano Panichella currently speaks three languages: Italian (mather tongue), English (\textit{C1}) and German (\textit{A2.2}). He is actively studying German. Specifically, he started to study the German language in the last years. Thus, followed two courses at the UZH in the past. He recently followed, during the fall semester 2020, a course at ZHAW for reaching the B1 German level and successfully passed the final evaluation exam. He plans to continue his studies to reach the B2 level during the spring semester of 2022, to be able to reach a C1 level during the year 2023.


%\halfblankline
%\blankline



\end{document}
