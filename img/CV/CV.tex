\documentclass[10pt]{article}
\usepackage{calc}
\usepackage{graphicx} 
\reversemarginpar
   
\usepackage[paper=letterpaper, 
            marginparwidth=1.2in, 
            marginparsep=.05in, 
            margin=1in,
            includemp]{geometry}

\setlength{\parindent}{0in}

\usepackage[shortlabels]{enumitem}
% ============================================================
%:Markup macros for proof-reading
\usepackage{ifthen}
\usepackage[normalem]{ulem} % for \sout
\usepackage{xcolor}
\newcommand{\ra}{$\rightarrow$}
\newboolean{showedits}
\setboolean{showedits}{true} % toggle to show or hide edits
%\setboolean{showedits}{false} % toggle to show or hide edits
\ifthenelse{\boolean{showedits}}
{
	\newcommand{\meh}[1]{\textcolor{red}{\uwave{#1}}} % please rephrase
	\newcommand{\ins}[1]{\textcolor{blue}{\uline{#1}}} % please insert
	\newcommand{\del}[1]{\textcolor{red}{\sout{#1}}} % please delete
	\newcommand{\chg}[2]{\textcolor{red}{\sout{#1}}{\ra}\textcolor{blue}{\uline{#2}}} % please change
	\newcommand{\nbe}[3]{
		{\colorbox{#3}{\bfseries\sffamily\scriptsize\textcolor{white}{#1}}}
		{\textcolor{#3}{\sf\small$\blacktriangleright$\textit{#2}$\blacktriangleleft$}}}
}{
	\newcommand{\meh}[1]{#1} % please rephrase
	\newcommand{\ins}[1]{#1} % please insert
	\newcommand{\del}[1]{} % please delete
	\newcommand{\chg}[2]{#2}
	\newcommand{\nbe}[3]{}
}
%
\newcommand\rA[1]{\nbe{Reviewer A}{#1}{cyan}}
\newcommand\rB[1]{\nbe{Reviewer B}{#1}{olive}}
\newcommand\rC[1]{\nbe{Reviewer C}{#1}{magenta}}
\newcommand\ANS[1]{\nbe{Response}{#1}{teal}}
% ============================================================
%:Put edit comments in a really ugly standout display
%\usepackage{ifthen}
\usepackage{amssymb}
\newboolean{showcomments}
\setboolean{showcomments}{true}
%\setboolean{showcomments}{false}
\newcommand{\id}[1]{$-$Id: scgPaper.tex 32478 2010-04-29 09:11:32Z oscar $-$}
\newcommand{\yellowbox}[1]{\fcolorbox{gray}{yellow}{\bfseries\sffamily\scriptsize#1}}
\newcommand{\triangles}[1]{{\sf\small$\blacktriangleright$\textit{#1}$\blacktriangleleft$}}
\ifthenelse{\boolean{showcomments}}
%{\newcommand{\nb}[2]{{\yellowbox{#1}\triangles{#2}}}
{\newcommand{\nbc}[3]{
 {\colorbox{#3}{\bfseries\sffamily\scriptsize\textcolor{white}{#1}}}
 {\textcolor{#3}{\sf\small$\blacktriangleright$\textit{#2}$\blacktriangleleft$}}}
 \newcommand{\version}{\emph{\scriptsize\id}}}
{\newcommand{\nbc}[3]{}
 \newcommand{\version}{}}
\newcommand{\nb}[2]{\nbc{#1}{#2}{orange}}
\newcommand{\here}{\yellowbox{$\Rightarrow$ CONTINUE HERE $\Leftarrow$}}
\newcommand\rev[2]{\nb{TODO (rev #1)}{#2}} % reviewer comments
\newcommand\fix[1]{\nb{FIX}{#1}}
\newcommand\todo[1]{\nb{TO DO}{#1}}
\newcommand\on[1]{\nbc{ON}{#1}{red}} % add more author macros here
%\newcommand\XXX[1]{\nbc{XXX}{#1}{blue}}
%\newcommand\XXX[1]{\nbc{XXX}{#1}{brown}}
%\newcommand\XXX[1]{\nbc{XXX}{#1}{cyan}}
%\newcommand\XXX[1]{\nbc{XXX}{#1}{darkgray}}
%\newcommand\XXX[1]{\nbc{XXX}{#1}{gray}}
%\newcommand\XXX[1]{\nbc{XXX}{#1}{magenta}}
%\newcommand\XXX[1]{\nbc{XXX}{#1}{olive}}
%\newcommand\XXX[1]{\nbc{XXX}{#1}{orange}}
%\newcommand\XXX[1]{\nbc{XXX}{#1}{purple}}
%\newcommand\XXX[1]{\nbc{XXX}{#1}{red}}
%\newcommand\XXX[1]{\nbc{XXX}{#1}{teal}}
%\newcommand\XXX[1]{\nbc{XXX}{#1}{violet}}
% ============================================================


\makeatletter
\newlength{\bibhang}
\setlength{\bibhang}{1em}
\newlength{\bibsep}
 {\@listi \global\bibsep\itemsep \global\advance\bibsep by\parsep}
\newlist{bibsection}{itemize}{3}
\setlist[bibsection]{label=,leftmargin=\bibhang,%
        itemindent=-\bibhang,
        itemsep=\bibsep,parsep=\z@,partopsep=0pt,
        topsep=0pt}
\newlist{bibenum}{enumerate}{3}
\setlist[bibenum]{label=[\arabic*],resume,leftmargin={\bibhang+\widthof{[999]}},%
        itemindent=-\bibhang,
        itemsep=\bibsep,parsep=\z@,partopsep=0pt,
        topsep=0pt}
\let\oldendbibenum\endbibenum
\def\endbibenum{\oldendbibenum\vspace{-.6\baselineskip}}
\let\oldendbibsection\endbibsection
\def\endbibsection{\oldendbibsection\vspace{-.6\baselineskip}}
\makeatother


\usepackage{fancyhdr,lastpage}
\pagestyle{fancy}
%\pagestyle{empty}      % Uncomment this to get rid of page numbers
\fancyhf{}\renewcommand{\headrulewidth}{0pt}
\fancyfootoffset{\marginparsep+\marginparwidth}
\newlength{\footpageshift}
\setlength{\footpageshift}
          {0.5\textwidth+0.5\marginparsep+0.5\marginparwidth-2in}
\lfoot{\hspace{\footpageshift}%
       \parbox{4in}{\, \hfill %
                    \arabic{page} of \protect\pageref*{LastPage} % +LP
%                    \arabic{page}                               % -LP
                    \hfill \,}}

% Finally, give us PDF bookmarks
\usepackage{color,hyperref}
\definecolor{darkblue}{rgb}{0.0,0.0,0.3}
\hypersetup{colorlinks,breaklinks,
            linkcolor=darkblue,urlcolor=darkblue,
            anchorcolor=darkblue,citecolor=darkblue}


\newcommand{\makeheading}[2][]%
        {\hspace*{-\marginparsep minus \marginparwidth}%
         \begin{minipage}[t]{\textwidth+\marginparwidth+\marginparsep}%
             {\large \bfseries #2 \hfill #1}\\[-0.15\baselineskip]%
                 \rule{\columnwidth}{1pt}%
         \end{minipage}}

\renewcommand{\section}[1]{\pagebreak[3]%
    \vspace{1.3\baselineskip}%
    \phantomsection\addcontentsline{toc}{section}{#1}%
    \noindent\llap{\scshape\smash{\parbox[t]{\marginparwidth}{\hyphenpenalty=10000\raggedright #1}}}%
    \vspace{-\baselineskip}\par}

\newcommand*\fixendlist[1]{%
    \expandafter\let\csname preFixEndListend#1\expandafter\endcsname\csname end#1\endcsname
    \expandafter\def\csname end#1\endcsname{\csname preFixEndListend#1\endcsname\vspace{-0.6\baselineskip}}}

\let\originalItem\item
\newcommand*\fixouterlist[1]{%
    \expandafter\let\csname preFixOuterList#1\expandafter\endcsname\csname #1\endcsname
    \expandafter\def\csname #1\endcsname{\csname preFixOuterList#1\endcsname\let\oldItem\item\def\item{\pagebreak[2]\oldItem}}
    \expandafter\let\csname preFixOuterListend#1\expandafter\endcsname\csname end#1\endcsname
    \expandafter\def\csname end#1\endcsname{\let\item\oldItem\csname preFixOuterListend#1\endcsname}}
\newcommand*\fixinnerlist[1]{%
    \expandafter\let\csname preFixInnerList#1\expandafter\endcsname\csname #1\endcsname
    \expandafter\def\csname #1\endcsname{\let\oldItem\item\let\item\originalItem\csname preFixInnerList#1\endcsname}
    \expandafter\let\csname preFixInnerListend#1\expandafter\endcsname\csname end#1\endcsname
    \expandafter\def\csname end#1\endcsname{\csname preFixInnerListend#1\endcsname\let\item\oldItem}}
 
\newlist{outerlist}{itemize}{3}
    \setlist[outerlist]{label=\enskip\textbullet,leftmargin=*}
    \fixendlist{outerlist}
    \fixouterlist{outerlist}

\newlist{lonelist}{itemize}{3}
    \setlist[lonelist]{label=\enskip\textbullet,leftmargin=*,partopsep=0pt,topsep=0pt}
    \fixendlist{lonelist}
    \fixouterlist{lonelist}

\newlist{innerlist}{itemize}{3}
    \setlist[innerlist]{label=\enskip\textbullet,leftmargin=*,parsep=0pt,itemsep=0pt,topsep=0pt,partopsep=0pt}
    \fixinnerlist{innerlist}

\newlist{loneinnerlist}{itemize}{3}
    \setlist[loneinnerlist]{label=\enskip\textbullet,leftmargin=*,parsep=0pt,itemsep=0pt,topsep=0pt,partopsep=0pt}
    \fixendlist{loneinnerlist}
    \fixinnerlist{loneinnerlist}

\newcommand{\blankline}{\quad\pagebreak[3]}
\newcommand{\halfblankline}{\quad\vspace{-0.5\baselineskip}\pagebreak[3]}

\newcommand\doilink[1]{\href{http://dx.doi.org/#1}{#1}}
\newcommand\doi[1]{doi:\doilink{#1}}

\providecommand*\url[1]{\href{#1}{#1}}
\renewcommand*\url[1]{\href{#1}{\texttt{#1}}}
\providecommand*\email[1]{\href{mailto:#1}{#1}}
\providecommand\BibTeX{{B\kern-.05em{\sc i\kern-.025em b}\kern-.08em
    \TeX}}
\providecommand\Matlab{\textsc{Matlab}}
\hyphenation{bio-mim-ic-ry bio-in-spi-ra-tion re-us-a-ble pro-vid-er}

\begin{document}
\makeheading{Sebastiano Panichella - Curriculum vitae}

\newlength{\rcollength}\setlength{\rcollength}{1.85in}%
\newlength{\spacewidth}\setlength{\spacewidth}{20pt}
\newcommand\spacechar{$|$}


\section{Biographical Sketch}

Sebastiano Panichella is a passionate  Computer Science Researcher (permanent position) at Zurich University of Applied Science (ZHAW), leading research in  Software Engineering (SE), cloud computing (CC), and Data Science (DS) research fields. \\


%Sebastiano Panichella received (cum laude) the Laurea in Computer Science from the University of Salerno (Italy) in December 2010 defending a thesis on IR-based Traceability Recovery.%, advised by Prof. Andrea De Lucia. 
He received the PhD in Computer Science from the University of Sannio (Department of Engineering) in  July 18th 2014 defending the thesis entitled  \href{http://dx.doi.org/10.1109/ICSM.2015.7332519}{\textit{``Supporting Newcomers in Open Source Software Development Projects"}}. \\
Previously he was postdoc at University of Zurich (\textit{\textbf{01-11-2014 - 19-08-2018}}) working in the \href{http://www.ifi.uzh.ch/seal.html}{\textit{SEAL Lab}} of \textbf{Prof. Harald Gall}.  During the postdoctoral experience, Dr. Panichella entirely wrote a proposal that was awarded (Sebastiano figured as co-applicant with Prof. Gall) by the Swiss National Science Foundation, i.e., the project SURF-MobileAppsData SNF -- No. 200021$-$166275-- (current results of the projects are available on-line \footnote{http://www.ifi.uzh.ch/en/seal/people/panichella/SNF-Projects.html}), which funded his research collaboration with the UZH (since 2016), on mobile computing and mobile testing, and two PhD Students. During the  experience as postdoc in the SEAL group he  investigated further SE research fields  such as Mobile Computing, Continuous Delivery and Continuous integration, and the  new line of research related to the use of Summarization Techniques for Code, Changes and Testing. 
\\\\
His main \textbf{research goal} is to conduct industrial research, involving both industrial and academic collaborations, to sustain
the Internet of Things (IoT) vision, where future smart cities will be characterized by millions of smart systems (e.g., cyber-physical
systems) connected over the internet, controlled by complex embedded software implemented for the cloud.\\
His  \textbf{research interests} are in the domain of Software Engineering (SE), cloud computing (CC), and Data Science (DS): DevOps (e.g., Continuous Delivery, Continuous integration), Machine learning applied to SE, Software maintenance and evolution (with particular focus on Cloud, mobile, AI-based, and Cyber-physical applications). Moreover, he is promoting DS research on \href{https://doi.org/10.1109/VST.2018.8327148}{\textit{Summarization Techniques for Code, Changes, and Testing}}. He is a \textbf{member of IEEE/ACM}. 
\vspace{1mm}

He authored or co-authored around \textbf{eighty} (considering also demonstration, dataset 
  and poster) papers appeared in International Conferences and Journals (26 of them published during the postdoctoral experience at the SEAL lab). His research projects involved relevant industrial companies (e.g., ING NEDERLAND, Sony Mobile Communication, SIEMENS, GVM, etc.) and their extensions will involve further industrial organizations and open source projects. He serves and has served as \href{https://spanichella.github.io/\#services}{program committee member} of various international conference (e.g., ICSE, SBST, ASE, ICPC, ICSME, SANER, MSR, SEAA) and as \href{https://spanichella.github.io/\#services}{reviewer for various international journals} (e.g., TSE, TOSEM, EMSE, JSS, IST, JSEP) in the fields of software engineering and evolutionary computation. He is currently Editorial Board Member of the \textit{Journal of Software: evolution and process} (JSEP) and in was recently \href{https://spanichella.github.io/\#services}{Lead (or Co-lead) Guest editor} of special issues at EMSE, JSEP, SCP and IST journals.\\


\textbf{Recent Achievements of Sebastiano Panichella:}

\vspace{-2.5mm}
\begin{itemize}
  \item According to the [Results reported by the JSS journal] Sebastiano Panichella was selected in 2021, according to the results reported by the JSS journal \footnote{https://www.sciencedirect.com/science/article/abs/pii/S0164121221001266}, as one of the \textbf{top-20 Most impactful SE researchers Worldwide in Software Engineering (SE)}. 
  \item According to the [Results reported by the JSS journal] Sebastiano Panichella was selected in 2019, according to the results reported by the JSS journal \footnote{https://www.sciencedirect.com/science/article/pii/S0164121218302334}, as one of the \textbf{top-20 Most Active Early Stage Researchers Worldwide in Software Engineering (SE)}. 
\vspace{-1.5mm}
  \item The paper [Sebastiano Panichella, Andrea Di Sorbo, Emitza Guzman, Corrado Aaron Visaggio, Gerardo Canfora, Harald C. Gall: How can I improve my app? Classifying user reviews for software maintenance and evolution. ICSME 2015: 281-290], which originated the idea behind this SNF project, is one of the \textbf{most cited papers of ICMSE 2015} (as reported in Google scholar), with over 400 citations in around 6-7 years.
  \item The research proposal submitted to the H2020 grant called COSMOS: ``DevOps for Complex Cyber-physical Systems" was selected for funding in 2021.
  \item The research proposal submitted to the Innosuisse called ``ARIES: Exploiting User Journeys and Testing Automation for Supporting Efficient Energy Service Platforms" was selected for funding in 2021.
\vspace{-1.5mm}
  \item The paper ICPC wrote during the bachelor studies of Dr. Panichella-[Giovanni Capobianco, Andrea De Lucia, Rocco Oliveto, Annibale Panichella, Sebastiano Panichella: On the role of the nouns in IR-based traceability recovery. ICPC 2009: 148-157] is \textbf{among the most influential papers of ICPC in the last decade [period 2009-2019]}.
  \vspace{-1.5mm}
\end{itemize}

\section{Contact Information}
%\textit{Address:} Department of Computer Engineering\\ \href{http://www.ing.unisannio.it/}{University of Sannio}\\
%RCOST- Palazzo ex Poste, Via Traiano\\
%82100 Benevento (Italy).\\\\
%\textit{Mobile:} +39 3881969673\\
%\textit{Tel.: +39 0824 305539} \\
%\textit{E-mail:} \email{spanichella@gmail.com}\\
%\textit{Home Page:} \href{http://www.ing.unisannio.it/spanichella}{www.ing.unisannio.it/spanichella}
%\textit{Address:} Department of Computer Engineering\\ \href{http://www.ing.unisannio.it/}{University of Sannio}\\
%RCOST- Palazzo ex Poste, Via Traiano\\
%82100 Benevento (Italy).\\\\
%\textit{Mobile:} +39 3881969673\\
%\textit{Tel.: +39 0824 305539} \\
%\textit{E-mail:} \email{spanichella@gmail.com}\\
%\textit{Home Page:} \href{http://www.ing.unisannio.it/spanichella}{www.ing.unisannio.it/spanichella}
\textit{Address:} \href{https://www.zhaw.ch/en/engineering/}{Zurich University of Applied Science (ZHAW)}\\ 
Obere Kirchgasse 2, 8400 Winterthur, Switzerland.\\ 
%\textit{Tel.: +39 0824 305539} \\
\textit{Tel.: +41 (0) 58 934 41 56} \\
\textit{E-mail:} \email{panc@zhaw.ch} (or alternatively \email{spanichella@gmail.com})\\
\textit{Home Page:} \href{https://spanichella.github.io/}{https://spanichella.github.io/}\\
\textit{Google Scholar Ref:}\\ \url{https://scholar.google.it/citations?user=HiNuBFgAAAAJ\&hl=en\&oi=ao}\\
\textit{Short CV:} \href{https://spanichella.github.io/img/CV.pdf}{https://spanichella.github.io/img/CV-short.pdf}\\
\textit{Detailed CV:} \href{https://spanichella.github.io/img/CV.pdf}{https://spanichella.github.io/img/CV.pdf}\\
\blankline

\section{Research Interests}

%\on{This section should say what *you* are doing in these fields, rather than just defining the fields!}

\textbf{Cyber-physical systems (CPSs) development}. Much of the increasing complexity of ICT systems is being driven by the more distributed and heterogeneous nature of these systems, with Cyber-Physical Systems accounting for an increasing portion of Software Ecosystems. This basic premise underpins the research conducted by Dr. Panichella in the COSMOS H2020 and the ARIES Innosuisse projects, which focuses on blending best practices DevOps solutions with the development processes used in the CPS context: this will enable the CPS world to deliver software more rapidly and result in more secure and trustworthy systems. COSMOS brings together a balanced consortium of big industry, SMEs and academics which will develop enhanced DevOps pipelines which target development of CPS software. The COSMOS CPS pipelines will be validated against several use cases provided by industrial partners representing healthcare, avionics, automotive, micromobility, utility and railway sectors. These will act as reference use cases when promoting the technology amongst Open Source and standardization communities. \\
    More information about the COSMOS H2020 project can be found at \\\url{https://www.cosmos-devops.org/}. \\More information about the ARIES Innosuisse project can be found at \\\url{https://aries-devops.ch/}.\\\\

\textbf{Machine Learning and Genetic Algorithms}
\blankline\\\\
Machine learning (ML) and Genetic Algorithms (GA) deals with the issue of how to build computer programs that improve their performance at some tasks through experience. ML and Genetic algorithms have proven to be of great practical value in a variety of application domains. Not surprisingly, the field of software engineering turns out to be a fertile ground where many software development and maintenance tasks could be formulated as learning problems and approached in terms of learning algorithms. 
   \textbf{Work in progress.} Dr. Panichella investigated the potential of using ML and Genetic Algorithms for solving SE problems. He started to study them during the PhD studies. Examples of the successful application of  ML and genetic algorithms to SE problems by Panichella are  bug prediction, code (and code change) prediction  \ref{C9}\ref{J1}\ref{J3}\ref{C24},  
 prioritization or clustering of user reviews (in the context of mobile apps) \ref{C2}\ref{C3}\ref{C5}\ref{C7}\ref{C8}\ref{C14}, test case generation \ref{C10}\ref{J05}, etc..  Recent and current research directions in this topic are toward experimenting customized solutions based on ML and Genetic Algorithms for enhancing traditional testing approaches and GUI testing processes \ref{Cm3}, identifying class comment types in multi-language projects \ref{J15}, supporting qualitative characterization and automated prediction of issue labels in Github \ref{Cm11}\ref{J14}\ref{J11},  monitoring vulnerability-proneness of Google Play Apps \ref{J12}.
  \\
   \\

\textbf{Continuous Delivery \& Testing Automation}
\blankline\\\\
Continuous delivery (CD) is a software engineering approach in which teams produce software in short cycles, ensuring that the software can be reliably released at any time. It aims at building, testing, and releasing software faster and more frequently. The approach helps reduce the cost, time, and risk of delivering changes by allowing for more incremental updates to applications in production. A straightforward and repeatable deployment process is important for continuous delivery. Continuous Integration (CI) consists in a specific stage of CD process where team members integrate their work in an automatic manner, which allows a fast building, testing, and releasing of software, leading to multiple integrations per day. Researchers in this field have as main focus the development of recommender systems able to provide suggestions and automated support to developers and testers during Continuous Integration activities. 
   \textbf{Work in progress.}  
   Dr. Panichella is very interested in investigate and overcome contemporary limitations of DevOps (e.g., continuous delivery and continuous integration) practices and tools for complex systems (e.g., Cloud and Cyber-physical systems). 
   In the context of CI Dr. Panichella is currently conducting empirical work to understand the problems that developers face when integrating new changes in the code base \ref{C0}\ref{Cm3}. The main focus is the development of recommender systems able to provide suggestions to developers and testers during Continuous Integration activities. In recent work he also investigated strategies to optimize test case generation in CI pipelines\ref{J05}\ref{J02},  contemporary bad practices affecting CI adoption \ref{J06}, technical debt analysis for Serverless \ref{J10},   the cloudification perspectives of search-based software testing \ref{Cm9}\label{Cm15}, approaches to measure structural coupling for microservices \ref{Cm20}, and how developers engage with static analysis tools in different development contexts (i.e., Code Review, CI, local development) \ref{J03}\ref{J04}\ref{J08}.
   On going research concerns branch coverage prediction in automated testing \ref{J02},  improving the readability of automatically generated Tests\ref{Cm16}, test smells in automatically generated tests\ref{Cm17}, and exploring the integration of user Feedback in Automated Testing of Android Applications \ref{Cm3}.
    \\

\textbf{Empirical Software Engineering}
\blankline\\\\
Empirical software engineering is a sub-domain of software engineering focusing on experiments on software systems (software products, processes, and resources). It is interested in devising experiments on software, in collecting data from these experiments, and in devising laws and theories from this data. Proponents of experimental software engineering advocate that the nature of software is such that we can advance the knowledge on software through experiments only. The scientific method suggests a cycle of observations, laws, and theories to advance science. Empirical software engineering applies this method to software.
   \textbf{Work in progress.} In past work Dr. Panichella performed empirical studies to understand (i) how   OSS communities upgrades dependencies \ref{J2}\ref{C21};   
   (ii) to what extent static analysis tools help developers with code reviews \ref{C16};  (iii)  how developers' collaborations identified from different sources vary when they are mined from different sources \ref{C17};  (iv) how the evolution of emerging collaborations relates to code changes \ref{C20}; (v) comment evolution and practices in Pharo Smalltalk \ref{J13}; or (vi) to study
   the behaviour of developers during maintenance tasks or pull requests development (e.g., while they modify existing features or fix a bug) by analyzing their navigation patterns  \ref{C22}\ref{Cm12}. Currently Dr. Panichella is focusing his attention in performing empirical work to understand possible ways to measure and foster developer  productivity during testing \ref{C10}, maintenance \ref{C22} and code reviewing tasks  \ref{C16} as well as investigating how developers discuss about code comments in social media \ref{Cm23} or how do communities in developer interaction networks align with Subsystem Developer Teams\ref{Cm19}.
   \\
   
\textbf{Mining Software Repositories \& User Feedback Analysis}
\blankline \\\\
Software repositories such as source control systems, archived communications between project personnel, and defect tracking systems are used to help manage the progress of software projects. Software practitioners and researchers are recognizing the benefits of mining this information to support the maintenance of software systems, improve software design/reuse, and empirically validate novel ideas and techniques. Research is now proceeding to uncover the ways in which mining these repositories can help to understand software development and software evolution, to support predictions about software development, and to exploit this knowledge concretely in planning future development. The Mining Software Repositories (MSR) field analyzes the rich data available in software repositories to uncover interesting and actionable information about software systems and projects.
   \textbf{Work in progress.}  In past work Panichella focused his attention in mining software repository to build recommender systems for supporting developers during maintenance and program comprehension tasks. For instance, he conceived tools for  (i) enabling the automatic re-documentation of existing systems \ref{C19} \ref{C27};  (ii) summarizing software artifacts \ref{Cm7} \ref{C26} \ref{J5};  (iii) or profiling developers or experts in OSS projects \ref{C12}\ref{C17}\ref{C18}\ref{C20}\ref{C23}\ref{C25}. 
   Recently Dr. Panichella  focused his attention in designing and developing tools to help  developers digest the huge amount of feedback they receive from users on a daily basis, transforming user reviews into maintenance tasks (fixing issues or building features) \ref{C8}\ref{C2}\ref{C3}\ref{C5}\ref{C7}\ref{J07}\ref{Cm14}\ref{Cm4}\ref{Cm1}; tools  for multi-source analysis based on unstructured data \ref{Cm17}
  Dr. Panichella is also focusing on studies investigating the criticality of User reported issues through their relations with app Rating \ref{J09}.      
   More in general, he is interested to conceive tools to support developers in evolving modern software applications \ref{C8}\ref{C14}. \\
   
\textbf{(Modern) Code Review}
\blankline\\\\
Peer code review, a manual inspection of source code by developers other than the author, is recognized as a valuable tool for reducing software defects and improving the quality of software projects. In 1976, Fagan formalized a highly structured process for code reviewing, based on line-by-line group reviews, done in extended meetings--code inspections. Over the years, researchers provided evidence on code inspection benefits, especially in terms of defect finding, but the cumbersome, time-consuming, and synchronous nature of this approach hinders its universal adoption in practice. Nowadays, many organizations are adopting more lightweight code review practices to limit the inefficiencies of inspections. In particular, there is a clear trend toward the usage of tools specifically developed to support code review. Modern code reviews are (1) informal (in contrast to Fagan-style), (2) tool-based, and (3) occurs regularly in practice nowadays, for example at companies such as Microsoft, Google, Facebook, and in other companies and OSS projects. %The growth in usage of the modern code review process raises many questions.
   \textbf{Work in progress.} The research focus of Panichella is to develop recommender systems able to (better) support developers during the code review process \ref{C16}. Hence recent effort was devoted in automatically configure static analysis tools during code review activities\ref{J16}\ref{J03} as well as investigation the relevant changes and automation needs of developers in modern code review \ref{J08}. 
   \\

\textbf{Textual analysis in SE}
\blankline\\\\
Textual analysis can be described as the examination of a text in which an educated guess is formed as to the most likely interpretations that might be made of that text. It is where the researcher must decentre the text to reconstruct it, working back through the narrative mediations of form, appearance, rhetoric, and style to uncover the underlying social and historical processes, the metalanguage that guided the production. It is suggested that textual analysis can cover four main underlying constructs: language and meaning, ideology, ideology and myth, and historicity. In this sense, textual analysis is a methodology: a way of gathering and analysing information in academic research (Mckee, A 2001).
\textbf{Work in progress.} Panichella  studied text analysis approaches since his bachelor
and master studies and was always fascinated by the great usability of Natural Language
Processing (NLP) and Information Retrieval (IR) tools and techniques for solving several
practical problems. He adopted such techniques in several work during his PhD and also
during the postdoctoral experience. He is currently learning new techniques and tools
based on Textual Analysis      (e.g. WORD2VEC) and neural networks techniques \ref{C1}. He 
also proposed an NLP-based tools \ref{Cm22} for software artifacts analysis to explore the natural language structures in software informal documentation \ref{J04} or to detect inconsistencies between documentation and code \ref{J01}.\\
   
   \textbf{IR-based Traceability Recovery}
\blankline\\\\
Traceability has been defined as "the ability to describe and follow the life of an artefact (requirements, code, tests, models, reports, plans, etc.), in both a forwards and backwards direction". Thus, traceability links help software engineers to understand the relationships and dependencies among various software artefacts (requirements, code, tests, models, etc.) developed during the software lifecycle. The two main research topics related to the traceability management are event-based systems for traceability management and information retrieval based methods and tools supporting the software engineer in the traceability link recovery.\\
   \textbf{Work in progress.} In past work Panichella explored several enhancing strategies for improving IR-based Traceability Recovery approaches, most of them are based on (i) smoothing filters  \ref{J3}\ref{J4} and (ii)  NLP approaches \ref{C28}\ref{C29}\ref{C30}. Recently Panichella is focusing his effort in tracing link between data and software artifacts stored in modern software repositories \ref{C2}\ref{C3}\ref{C5}.
   \\

\section{Academic Appointments} 

Currently he is a (Permanent) Senior Research Associate at Zurich University of Applied Science (from \textit{\textbf{20-08-2018}}). Previously he was postdoc at University of Zurich (\textit{\textbf{01-11-2014 - 19-08-2018}}) working in the \href{http://www.ifi.uzh.ch/seal.html}{\textit{SEAL Lab}} of \textbf{Prof. Harald Gall}.  He is a \textbf{member of IEEE/ACM}. During the  experience as postdoc in the SEAL group he  investigated further  SE research fields  such as Mobile Computing, Continuous Delivery and Continuous integration, and the  new line of research related to the use of Summarization Techniques for Code, Changes and Testing. Currently His research interests include Mobile/Cloud Computing, IR-based Traceability Recovery, Textual Analysis, Machine Learning and Genetic Algorithms applied to SE problems, Continuous Delivery (with special attention to Continuous Integration Problems), Software maintenance and evolution and Empirical Software Engineering (with particular focus on Cloud Applications). Another topic that is also of his interest is Code Review,  indeed, he is currently working and advising students on research ideas aimed at automating the process of code inspection. His research was funded by a Swiss National Science Foundations project during the experience at the SEAL group of Prof. Gall.

\section{Academic Experience and History}

(Permanent) Senior Computer Science Researcher in Software Engineering (SE), cloud computing (CC), and Data Science (DS) at ZHAW (from 20-08-2018) and Part-time (External) Lecturer at the University of Zurich (from \textit{20-08-2018}).

\href{}{\textbf{University of Zurich}}, Switzerland\\

Postdoc at University of Zurich working in the  SEAL Lab of \textbf{Prof. Harald Gall}. Period \textit{\textbf{01-11-2014 - 19-08-2018}}.\\ 


\href{http://www.ing.unisannio.it/}{\textbf{University of Sannio}}, Italy
\begin{outerlist}
\item[] PhD.,
        \href{http://www.ing.unisannio.it/}
             {Computer Engineering}, July 2014
        \begin{innerlist}
        %\bigsqcup
        \item Thesis Title: \emph{\href{http://www.ifi.uzh.ch/seal/people/panichella/Dissertation_Sebastiano_Panichella.pdf}{\textit{``Supporting Newcomers in Open Source Software Development Projects"}}}
        %\item advisors:
         %     \href{www.gerardocanfora.net/‎}
                   %{Prof. Gerardo Canfora} and
          %    \href{www.rcost.unisannio.it/mdipenta/}
           %        {Prof. Massimiliano Di Penta}
        \item Thesis Topics: Supporting Developers, Mining of Software Repositories (Mailing lists, Issue trackers, Versioning Systems etc.)\\
 \end{innerlist}
\end{outerlist}

\href{http://www.unisa.it/}{\textbf{University of Salerno}}, Italy
\begin{outerlist}
\item[] M.S.,
        \href{http://www.unisa.it/facolta/scienze_mmffnn/index}
             {Computer Science}, December 2010
        \begin{innerlist}
        \item \emph{Magna cum Laude}
        \item Thesis Title: \emph{Improving IR-based Traceability Recovery Using Smoothing Filters}
        \item advisor:
              \href{http://www.dmi.unisa.it/people/delucia/www/}
                   {Prof. Andrea De Lucia}
        \item Thesis Topics: Software Engineering, Traceability Recovery, Textual Analysis\\
 \end{innerlist}
\end{outerlist}


\href{http://www.unimol.it/}{\textbf{University of Molise}}, Italy
\begin{outerlist}

\item[] B.S.,
        \href{http://www.unimol.it/pls/unimolise/v3_s2ew_consultazione.mostra_pagina?id_pagina=51098}
             {Computer Science}, October 2008
        \begin{innerlist}
        \item \emph{Magna cum Laude}
        \item Thesis Title: \emph{Improving IR-based traceability recovery via noun-based indexing of software artifacts}
        \item advisors:
              \href{http://docenti.unimol.it/index.php?u=giovanni.capobianco}
                   {Prof. Giovanni Capobianco},
              \href{http://www.distat.unimol.it/people/oliveto/Home.html}
                   {Dr Rocco Oliveto}
        \item Thesis Topics: Software Engineering, Traceability Management, Natural Language Processing (NLP)
        \end{innerlist}

\end{outerlist} 

%\newpage
\blankline

\section{Refereed Journal Publications}
In papers marked with (*)  the authors are listed in alphabetic order. %When such a rule is not followed, authors are listed by contribution.
\textbf{\\\underline{Journal Publications after the P.h.D.}}\\
\begin{bibenum}
\item \label{J16} Fiorella Zampetti, Saghan Mudbhari, Venera Arnaoudova, Massimiliano Di Penta, \underline{Sebastiano Panichella}, Giuliano Antoniol: Using Code Reviews to Automatically Configure Static Analysis Tools.    Empirical Software Engineering.  \doi{https://link.springer.com/article/10.1007/s10664-021-10076-4} 
\item \label{J15} Pooja Ruhal, \underline{Sebastiano Panichella}, Manuel Leuenberger, Andrea Di Sorbo, and Oscar Nierstrasz: How to Identify Class Comment Types? A Multi-language Approach for Class Comment Classification. Journal of Systems and Software.  \doi{https://doi.org/10.1016/j.jss.2021.111047}
\item \label{J14} \underline{Sebastiano Panichella}, Gerardo Canfora, and Andrea Di Sorbo: "Won't We Fix this Issue?" Qualitative Characterization and Automated Identification of Wontfix Issues on Github. Information and Software Technology Journal. \\ \doi{https://doi.org/10.1016/j.infsof.2021.106665} 
\item \label{J13}  Pooja Rani, \underline{Sebastiano Panichella}, Manuel Leuenberger, Mohammad Ghafari, Oscar Nierstrasz: What do class comments tell us? An investigation of comment evolution and practices in Pharo Smalltalk. Empirical Software Engineering. \\ \doi{https://doi.org/10.1007/s10664-021-09981-5} 
\item \label{J12}  Andrea Di Sorbo and \underline{Sebastiano Panichella}: Exposed! A Case Study on the Vulnerability-Proneness of Google Play Apps.    Empirical Software Engineering. \\ \doi{https://link.springer.com/article/10.1007/s10664-021-09978-0} 
\item \label{J11}  Rafael Kallis, Andrea Di Sorbo, Gerardo Canfora, \underline{Sebastiano Panichella}: Predicting Issue Types on GitHub.  Journal of Science of Computer Programming.   \\ \doi{https://doi.org/10.1016/j.scico.2020.102598} 
\item \label{J10}   Valentina Lenarduzzi, Jeremy Daly, Antonio Martini, \underline{Sebastiano Panichella}, Damian Andrew Tamburri: Towards a Technical Debt Conceptualization for Serverless Computing.   IEEE Software.   \\ \doi{https://ieeexplore.ieee.org/document/9222009} 
\item \label{J09}  Andrea Di Sorbo, Giovanni Grano, Aaron Visaggio and \underline{Sebastiano Panichella}:  Investigating the Criticality of User Reported Issues through their Relations with App Rating.   Journal of Software: Evolution and Process (JSEP) Journal. \\ \doi{https://onlinelibrary.wiley.com/doi/abs/10.1002/smr.2316}Z
\item \label{J08}  \underline{Sebastiano Panichella} and Nik Zaugg:  An Empirical Investigation of Relevant Changes and Automation Needs in Modern Code Review.   Empirical Software Engineering (EMSE) Journal. \\ \doi{https://link.springer.com/article/10.1007/s10664-019-09785-8} 
\item \label{J07} Yu Zhou, Yanqi Su, Taolue Chen, Zhiqiu Huang, Harald Gall, \underline{Sebastiano Panichella}:  User Review-Based Change File Localization for Mobile Applications.  Transactions on Software Engineering (TSE) Journal. \\ \doi{https://doi.org/10.1109/TSE.2020.2967383} 
\item \label{J06} Fiorella Zampetti, Carmine Vassallo, \underline{Sebastiano Panichella}, Gerardo Canfora, Harald Gall, Massimiliano Di Penta:  \textbf{An Empirical Characterization of Bad Practices in Continuous Integration}.  \emph{Empirical Software Engineering (EMSE)}. \\ \doi{https://doi.org/10.1109/TSE.2019.2946773}
\item \label{J05} Giovanni Grano, Christoph Laaber, Annibale Panichella, and \underline{Sebastiano Panichella}:  \textbf{Testing with Fewer Resources: An Adaptive Approach to Performance-Aware Test Case Generation}.  \emph{Transactions on Software Engineering (TSE)}.  \\ \doi{https://doi.org/10.1109/TSE.2019.2946773}\\
\item \label{J04} A. Sorbo, \underline{S. Panichella},  Aaron Visaggio, Di Massimiliano Di Penta, Canfora Gerardo, and Harald Gall. \textbf{Exploiting Natural Language Structures in Software Informal Documentation}. \emph{Transactions on Software Engineering (TSE)} 2019. \\ \doi{https://doi.org/10.1109/TSE.2019.2930519}\\
\item \label{J03} C. Vassallo, \underline{S. Panichella}, F. Palomba, S. Proksch, Harald Gall, Andy Zaidman. \textbf{How Developers Engage with Static Analysis Tools in Different Contexts}. \emph{Empirical Software Engineering (EMSE)} 2019.\\ \doi{https://doi.org/10.1007/s10664-019-09750-5}\\
    
\item \label{J02} G.Grano, T. Titov, \underline{S. Panichella}, H. Gall: \textbf{Branch Coverage Prediction in Automated Testing}.  \emph{Journal of Software: Evolution and Process (JSEP) 2019}. \\ \doi{https://doi.org/10.1016/j.infsof.2019.05.005}\\

\item \label{J0} C. Alexandru,  \underline{S. Panichella}, \underline{S. Proksch}, Harald Gall. \textbf{*Redundancy-free Analysis of Multi-revision Software Artifacts}. \emph{Empirical Software Engineering (EMSE)} 2019.\\
    \doi{http://doi.acm.org/10.1145/3276954.3276960}\\
      \item \label{J01}  Y. Zhou and 
   C. Wang and Y. Xin and T. Chen and \underline{S. Panichella} and H. Gall . \textbf{Automatic Detection and Repair Recommendation of Directive Defects in Java API Documentation}.  \emph{Transaction on Software Engineering 2018} - \\\url{https://ieeexplore.ieee.org/document/8478004}.

\item \label{J1} G. Canfora, A. De Lucia, M. Di Penta, R. Oliveto, A. Panichella, \underline{S. Panichella}. \textbf{*Defect Prediction as a Multi-Objective Optimization Problem}. \emph{Software Testing, Verification and Reliability (STVR)} 2015.\\
    \doi{10.1002/stvr.1570}\\
  \end{bibenum}
  
  \textbf{\\\underline{Journal Publications during the PhD study:}}\\
  \begin{bibenum}
    \item \label{J2} G. Bavota, G. Canfora, M. Di Penta, R. Oliveto, \underline{S. Panichella}. \textbf{*How the Apache Community Upgrades Dependencies}. \emph{Empirical Software Engineering (EMSE)} 2014.\\
             \doi{10.1007/s10664-014-9325-9}
             
    \item \label{J3}  A. De Lucia, M. Di Penta, R. Oliveto, A. Panichella, \underline{S. Panichella}. \textbf{*Applying a Smoothing Filter to Improve IR-based Traceability Recovery Processes: An Empirical Investigation}. \emph{Information and Software Technology (INFSOF)} 2012.\\
        \doi{10.1016/j.infsof.2012.08.002}

    \item \label{J5} A. De Lucia, M. Di Penta, R. Oliveto, A. Panichella, \underline{S. Panichella}. \textbf{*Labeling Source Code with Information Retrieval Methods: An Empirical Study}. \emph{Empirical Software Engineering (EMSE)} 2013.
                \doi{doi:10.1007/s10664-013-9285-5}
\end{bibenum} 

  \textbf{\\\underline{Journal Publications during the master study:}}\\
\begin{bibenum}
    \item \label{J4}  G. Capobianco, A. De Lucia, R. Oliveto, A. Panichella, \underline{S. Panichella}. \textbf{*Improving IR-based traceability recovery via noun-based indexing of software artifacts}. \emph{Journal of Software: Evolution and Process (JSE)} 2012.\\
        \doi{10.1002/smr.1564}
\end{bibenum} 
%\section{Submitted Journal Publications}
%\begin{bibenum}
%     \item     \underline{S. Panichella} Harald Gall. \textbf{Recommending Mentors in Open Source Projects to Grow-Up Long Term Contributors}. In: \emph{JSEP}\\
%\end{bibenum}

%\blankline   


 
\blankline

\section{Conference Publications}
In papers marked with (*)  the authors are listed in alphabetic order. %When such a rule is not followed, authors are listed by contribution.

\textbf{\\\underline{Conference/Wowkshop Publications after the Ph.D.:}}\\
\begin{bibenum}
 %\textit{Core RANK: B}.
  		\item \label{Cm24} Christian Birchler, Nicolas Ganz, Sajad Khatiri, Alessio Gambi and \underline{Sebastiano Panichella}:  \textbf{Cost-effective Simulation-based Test Selection in Self-driving Cars Software with SDC-Scissor.}  \emph{International Conference on Software Analysis, Evolution, and Reengineering 2022 (SANER)} - \textit{to Appear}. 
  		\item \label{Cm23} Pooja Ruhal, Mathias Birrer, \underline{Sebastiano Panichella}, Mohammad Ghafari, and Oscar Nierstrasz:  \textbf{What do Developers Discuss about Code Comments?}  \emph{International Working Conference on Source Code Analysis and Manipulation 2021 (SCAM)} - \textit{https://doi.org/10.1109/SCAM52516.2021.00027}.
 		\item \label{Cm22} Andrea Di Sorbo, Aaron Visaggio, Massimiliano Di Penta, Gerardo Canfora, \underline{Sebastiano Panichella}:  \textbf{An NLP-based Tool for Software Artifacts Analysis}  \emph{International Conference on Software Maintenance and Evolution} - \textit{https://doi.org/10.1109/ICSME52107.2021.00058}.
 		\item \label{Cm21} \underline{Sebastiano Panichella}, Alessio Gambi, Fiorella Zampetti, Vincenzo Riccio:  \textbf{SBST Tool Competition 2021}  \emph{International Conference on Software Engineering Workshops (ICSE 2021)} - \doi{10.1109/SBST52555.2021.00011}. 
       \item \label{Cm20} \underline{Sebastiano Panichella}, Mohammad Imranur Rahman, and Davide Taibi:  \textbf{Structural Coupling for Microservices}  \emph{International Conference on Cloud Computing and Services Science (CLOSER 2021)}\\ \textit{https://www.scitepress.org/Link.aspx}\doi{10.5220/0010481902800287}. 
       \item \label{Cm19} Usman Ashraf, Christoph Mayr-Dorn, Atif Mashkoor, Alexander Egyed, and \underline{Sebastiano Panichella}:  \textbf{Do Communities in Developer Interaction Networks align with Subsystem Developer Teams? An Empirical Study of Open Source Systems}  \emph{International Conference on Software and System Processes (ICSSP 2021)} - \textit{10.1109/ICSSP-ICGSE52873.2021.00016}.   
      \item \label{Cm17_2} Mathias Birrer, Pooja Ruhal, \underline{Sebastiano Panichella}, and Oscar Niestrasz:  \textbf{Makar: A Framework for Multi-source Studies based on Unstructured Data}  \emph{International Conference on Software Analysis, Evolution and Reengineering (SANER 2021)}\\ \textit{https://ieeexplore.ieee.org/document/9426063}.   
      \item \label{Cm17} Annibale Panichella, \underline{Sebastiano Panichella}, Gordon Fraser, Anand Ashok Sawant and Vincent Hellendoorn:  \textbf{Revisiting Test Smells in Automatically Generated Tests: Limitations, Pitfalls, and Opportunities}  \emph{International Conference on Software Maintenance and Evolution (ICSME 2020)} \\- \textit{https://ieeexplore.ieee.org/document/9240691}. 
      \item \label{Cm16} Devjeet Roy, Ziyi Zhang, Maggie Ma, Venera Arnaoudova, Annibale Panichella,  \underline{Sebastiano Panichella}, Danielle Gonzalez, Mehdi Mirakhorli:  \textbf{DeepTC-Enhancer: Improving the Readability of Automatically Generated Tests.}  \emph{IEEE/ACM International Conference on Automated Software Engineering} \\ \doi{https://doi.org/10.1145/3324884.3416622}. 
      \item \label{Cm15} Xavier Devroey, \underline{Sebastiano Panichella} and Alessio Gambi:  \textbf{Java Unit Testing Tool Competition-Eighth Round.}.  \emph{IEEE/ACM 42nd International Conference on Software Engineering Workshops (ICSE 2020)} \\- \textit{https://doi.org/10.1145/3387940.3392265}. 
    \item \label{Cm14} \underline{Sebastiano Panichella} and Marcela Ruiz:  \textbf{Requirements-Collector: Automating Requirements Specification from Elicitation Sessions and User Feedback}.  \emph{IEEE International Requirements Engineering Conference (RE'20).} \\\textit{https://doi.org/10.1109/RE48521.2020.00057}. 
      \item \label{Cm13} Usman Ashraf, Christoph Mayr-Dorn, Alexander Egyed, and \underline{Sebastiano Panichella}:  \textbf{A Mixed Graph-Relational Dataset of Socio-technical interactions in Open Source Systems}.  \emph{Mining Software Repositories (MSR 2020).} - \\\textit{https://doi.org/10.1145/3379597.3387492}. %\textit{Core RANK: B}.

      \item \label{Cm12}  Muhammad Ilyas Azeem, \underline{Sebastiano Panichella}, Andrea Di Sorbo, Alexander Serebrenik, and Qing Wang:  \textbf{Action-based Recommendation in Pull-request Development}.  \emph{International Conference on Software and System Processes (ICSSP2020)} - \\\textit{https://dl.acm.org/doi/10.1145/3379177.3388904}. %\textit{Core RANK: B}.

      \item \label{Cm11}  Rafael Kallis, Andrea Di Sorbo, Gerardo Canfora and \underline{Sebastiano Panichella}:   ICSE 2019 - To Appear. \textbf{Ticket Tagger: Machine Learning Driven Issue Classification}.  \emph{35th IEEE International Conference on Software Maintenance and Evolution (ICSME 2019)} - \\\textit{https://ieeexplore.ieee.org/iel7/8910135/8918933/08918993.pdf}. %\textit{Core RANK: B}.
      \item \label{Cm10}  Y. Zhou, C. Wang, Y. Xin, T. Chen, \underline{Sebastiano Panichella}, and H. Gall.:   ICSE 2019 - To Appear. \textbf{DRONE: A Tool to Detect and Repair Directive Defects in Java APIs Documentation}.  \emph{International Conference on Software Engineering, ICSE 2019} - \\\textit{https://ieeexplore.ieee.org/document/8802660}. %\textit{Core RANK: B}.
      
            \item \label{Cm9}  Diego Martin,  \underline{Sebastiano Panichella}. \textbf{The Cloudification Perspectives of Search-based Software Testing}.  \emph{The 12th Int. Workshop on Search-Based Software Testing, 2019} - \\\textit{https://ieeexplore.ieee.org/document/8812184}.

      \item \label{Cm8}  Carol V. Alexandru; Jose J. Merchante; \underline{Sebastiano Panichella}; Sebastian Proksch; Harald C. Gall; Gregorio Robles. \textbf{On the Usage of Pythonic Idioms.}.  \emph{Onward! 2018} - \\\textit{https://dl.acm.org/citation.cfm?id=3276960}. %\textit{Core RANK: B}.
 \item \label{Cm7}  \underline{S. Panichella}. \textbf{Summarization Techniques for Code, Change, Testing and User Feedback.}.  \emph{Proceedings of the  {IEEE} 25th International Conference on Software Analysis, Evolution
               and Reengineering}  (SANER 2018) - \\\textit{https://doi.org/10.1109/VST.2018.8327148}. %\textit{Core RANK: B}.
   \item \label{Cm6}  G. Grano, T. Titov, \underline{S. Panichella}, H. Gall. \textbf{How High Will It Be? Using Machine Learning Models to Predict Branch Coverage in Automated Testing}.  \emph{MaLTeSQuE}  (collocated with SANER 2018) - \\\textit{https://doi.org/10.1109/MALTESQUE.2018.8368454}. %\textit{Core RANK: B}.
  \item \label{Cm5}  L. Pelloni, G. Grano, A. Ciurumelea, \underline{S. Panichella}, F. Palomba, H. Gall. \textbf{BECLoMA: Augmenting Stack Traces with User Review Information.}.  \emph{Proceedings of the  {IEEE} 25th International Conference on Software Analysis, Evolution
               and Reengineering}  (SANER 2018) - \\\textit{https://doi.org/10.1109/SANER.2018.8330252}. %\textit{Core RANK: B}.
    \item \label{Cm4} A. Ciurumelea, \underline{S. Panichella}, H. Gall. \textbf{Automated User Reviews Analyser.}.  \emph{Proceedings of the  40th International Conference on Software Engineering}  (ICSE 2018) - \\\textit{http://doi.acm.org/10.1145/3183440.3194988}. %\textit{Core RANK: B}.

      \item \label{Cm3}  G. Grano, A. Ciurumelea, \underline{S. Panichella}, F. Palomba, H. Gall. \textbf{ Exploring the Integration of User Feedback in Automated Testing of Android Applications.}.  \emph{Proceedings of the  {IEEE} 25th International Conference on Software Analysis, Evolution
               and Reengineering}  (SANER 2018) - \\\textit{https://doi.org/10.1109/SANER.2018.8330198}. %\textit{Core RANK: B}.


      \item \label{Cm2}  C. Vassallo, \underline{S. Panichella}, F. Palomba, S. Proksch, A. Zaidman and H. Gall. \textbf{Context is King: The Developer Perspective on the Usage of Static Analysis Tools.}.  \emph{Proceedings of the  {IEEE} 25th International Conference on Software Analysis, Evolution
               and Reengineering}  (SANER 2018) - \\\textit{https://doi.org/10.1109/SANER.2018.8330195}. %\textit{Core RANK: B}.
      \item \label{Cm1}  G. Grano, A. Di Sorbo, F. Mercaldo, C. Visaggio, G. Canfora, \underline{S. Panichella}. \textbf{Android Apps and User Feedback: a Dataset for Software Evolution and Quality Improvement.}.  \emph{Proceedings of the  International Workshop on App Market Analytics}  (WAMA 2017). \\\textit{http://doi.acm.org/10.1145/3121264.3121266} 

      \item \label{C0}  C. Vassallo, G. Schermann, F. Zampetti, D. Romano, P. Leitner, A. Zaidman, M. di Penta, \underline{S. Panichella}.
         \textbf{A Tale of CI Build Failures: an Open Source and a Financial Organization Perspective.}.  \emph{Proceedings of the 33rd International Conference on Software Maintenance and Evolution}  (ICSME 2017). \textit{Core RANK: A}. \\\textit{https://doi.org/10.1109/ICSME.2017.67}
         
      \item \label{C1}  C. V. Alexandru, \underline{S. Panichella}, Harald Gall.
         \textbf{Replicating Parser Behavior using Neural Machine Translation}.  \emph{Proceedings of the 25th International Conference on Program Comprehension} (ICPC 2017). \textit{Core RANK: C}. \\\textit{https://doi.org/10.1109/ICPC.2017.11}
      \item \label{C2}  A. Di Sorbo,  \underline{S. Panichella}, C. V. Alexandru, C. A. Visaggio, G. Canfora.
         \textbf{SURF: Summarizer of User Reviews Feedback}.  Demonstrations Track of the  \emph{39th International Conference on Software Engineering} (ICSE 2017). \textit{Core RANK: A*}. \\\textit{https://doi.org/10.1109/ICSE-C.2017.5}

\item  \label{C3} F. Palomba, P. Salza, A. Ciurumelea,  \underline{S. Panichella}, H. Gall, F. Ferrucci,  A. De Lucia   \textbf{Recommending and Localizing Change Requests for Mobile Apps based on User Reviews}.    In: \emph{39th International Conference on Software Engineering} (ICSE 2017).  \textit{Core RANK: A*}. \\\textit{https://doi.org/10.1109/ICSE.2017.18}
       \item   \label{C5}    A. Ciurumelea, A. Schaufelbuhl, \underline{S. Panichella}, Harald Gall. \textbf{ Analyzing Reviews and Code of Mobile Apps for better Release Planning}. In: \emph{Proceedings of the 24th IEEE International Conference on Software Analysis, Evolution, and Reengineering (SANER) 2017. Klagenfurt, Austria}. \\\textit{https://doi.org/10.1109/SANER.2017.7884612} %\textit{Core RANK: B}.
  \item  \label{C4} Y. Zhou, R. Gu, T. Chen, Z. Huang,  \underline{S. Panichella}, H. Gall.   \textbf{Analyzing APIs Documentation and Code to Detect Directive Defects}.    In: \emph{39th International Conference on Software Engineering} (ICSE 2017).  \textit{Core RANK: A*}. \\\textit{https://doi.org/10.1109/ICSE.2017.11}
   \item   \label{C6}    C. Alexandru,  \underline{S. Panichella}, Harald Gall. \textbf{Reducing Redundancies in Multi-Revision Code Analysis}. In: \emph{24th IEEE International Conference on Software Analysis, Evolution, and Reengineering  (SANER) 2017. Klagenfurt, Austria}.  
                \\\textit{https://doi.org/10.1109/SANER.2017.7884617} 
              %\textit{Core RANK: B}.
              
    \item  \label{C7}  \underline{S. Panichella}, A. Di Sorbo, E. Guzman, C. Visaggio, G. Canfora, H. Gall.
         \textbf{ARdoc: App Reviews Development Oriented Classifier}.  In: \emph{24th ACM SIGSOFT International Symposium on the Foundations of Software Engineering}  will be held in Seattle, WA, USA.  \textit{Core RANK: A*}.
          \\\textit{http://doi.acm.org/10.1145/2950290.2983938}
  
      \item  \label{C8}  A. Di Sorbo,  \underline{S. Panichella}, C. V. Alexandru, J. Shimagaki, C. A. Visaggio, G. Canfora, H. Gall.
         \textbf{What Would Users Change in My App? Summarizing App
Reviews for Recommending Software Changes}.  In: \emph{24th ACM SIGSOFT International Symposium on the Foundations of Software Engineering}  will be held in Seattle, WA, USA.  \textit{Core RANK: A}. \\\textit{http://doi.acm.org/10.1145/2950290.2950299}

         \item  \label{C9}  A. Panichella, C. Alexandru,  \underline{S. Panichella}, A. Bacchelli, H. Gall. \textbf{A Search-based Training Algorithm for Cost-aware Defect Prediction}.  \emph{25th International Conference on Genetic Algorithms (ICGA) and the 21st Annual Genetic Programming Conference (GP) (GECCO 2016)}.  Denver, Colorado, USA.  \textit{Core RANK: A}. \\\textit{http://doi.acm.org/10.1145/2908812.2908938}
        
    \item  \label{C10}   \underline{S. Panichella}, A. Panichella, M. Bella, A. Zaidman, H. Gall. \textbf{The impact of test case summaries on bug fixing performance: An empirical investigation}. In: \emph{Proceedings of the 38th International Conference on Software Engineering} (ICSE 2016), Austin, TX.  \textit{Core RANK: A*}. \\\textit{http://doi.acm.org/10.1145/2884781.2884847}

    \item   \label{C11}   A. Di Sorbo, \underline{S. Panichella}, C. Visaggio, M. Di Penta, G. Canfora,  H. Gall. . \textbf{DECA: Development Emails Content Analyzer}. In: \emph{Proceedings of the 38th International Conference on Software Engineering} (ICSE 2016), Austin, TX.  \textit{Core RANK: A*}. \\\textit{http://doi.acm.org/10.1145/2889160.2889170}

    \item  \label{C12}   \underline{S. Panichella}. \textbf{Supporting Newcomers in Software Development Projects}. In: \emph{Proceedings of the 31st International Conference on Software Maintenance and Evolution} (ICSME 2015). Bremen, Germany.  \textit{Core RANK: A}. \\\textit{https://doi.org/10.1109/ICSM.2015.7332519}

        \item  \label{C13}  A. Di Sorbo, \underline{S. Panichella}, C. Visaggio, M. Di Penta, G. Canfora,  H. Gall. \textbf{Development Emails Content Analyzer: Intention Mining in Developer Discussions}. In: \emph{30th international conference on Automated Software Engineering} (ASE 2015).  Lincoln, Nebraska.  \textit{Core RANK: A}. \\\textit{https://doi.org/10.1109/ASE.2015.12}

    \item  \label{C14}   \underline{S. Panichella}, A. Di Sorbo, E. Guzman, C. Visaggio, G. Canfora, H. Gall. \textbf{How Can I Improve My App? Classifying User Reviews for Software Maintenance and Evolution}. In: \emph{Proceedings of the 31st International Conference on Software Maintenance and Evolution} (ICSME 2015). Bremen, Germany.  \textit{Core RANK: A}. \\\textit{https://doi.org/10.1109/ICSM.2015.7332474}

    \item  \label{C15}  G. Schermann, M. Brandtner,  \underline{S. Panichella}, P. Leitner,  H. Gall. \textbf{Discovering Loners and Phantoms in Commit and Issue Data}. In: \emph{Proceedings of the 37th International Conference on Program Comprehension} (ICPC 2015). Firenze, Italy.  \textit{Core RANK: C}. \\\textit{https://doi.org/10.1109/ICPC.2015.10}


    \item   \label{C16}  \underline{S. Panichella}, V. Arnaoudova, M. Di Penta, G. Antoniol. \textbf{Would Static Analysis Tools Help Developers with Code Reviews?}.  In: \emph{Proceedings of the 22nd International Conference on Software Analysis, Evolution and Reengineering} (SANER 2015). Montreal, Canada. \\\textit{https://doi.org/10.1109/SANER.2015.7081826}  %\textit{Core RANK: B}.

\end{bibenum}

\textbf{\\\underline{Conference Publications during the PhD experience:}}\\
\begin{bibenum}
    \item  \label{C17}  \underline{S. Panichella}, G. Bavota, M. Di Penta, G. Canfora, G. Antoniol. \textbf{How Developers' Collaborations Identified from Different Sources Tell us About Code Changes}. In: \emph{Proceedings of the 30th International Conference on Software Maintenance and Evolution} (ICSME 2014). Victoria, Canada.  \textit{Core RANK: A}. \\\textit{https://doi.org/10.1109/ICSME.2014.47}

    \item  \label{C18}  G. Bavota, \underline{S. Panichella}, N. Tsantalis, M. Di Penta, R. Oliveto, G. Canfora. \textbf{Recommending Refactorings based on Team Co-Maintenance Patterns.}. In: \emph{29th international conference on Automated Software Engineering} (ASE 2014). Vasteras, Sweden.  \textit{Core RANK: A}.
    \\\textit{https://doi.org/10.1109/ICSE.2017.18}

    \item  \label{C19}  C. Vassallo, \underline{S. Panichella}, G. Canfora, M. Di Penta. \textbf{CODES: mining sourCe cOde Descriptions from developErs diScussions}. In: \emph{Proceedings of the 36th International Conference on Program Comprehension} (ICPC 2014). Hyderabad, India.  \textit{Core RANK: C}. \\\textit{http://doi.acm.org/10.1145/2597008.2597799}

    \item  \label{C20}  \underline{S. Panichella}, G. Canfora, M. Di Penta, R. Oliveto. \textbf{How the Evolution of Emerging Collaborations Relates to Code Changes: an Empirical Study}. In: \emph{Proceedings of the 36th International Conference on Program Comprehension} (ICPC 2014). Hyderabad, India.  \textit{Core RANK: C}.
    \\\textit{http://doi.acm.org/10.1145/2597008.2597145}

    \item  \label{C21}  G. Bavota, G. Canfora, M. Di Penta, R. Oliveto, \underline{S. Panichella}. \textbf{*The Evolution of Project Inter-Dependencies in a Software Ecosystem: the Case of Apache}. In: \emph{Proceedings of the 29th International Conference on Software Maintenance} (ICSM 2013). Eindhoven, Netherlands.  \textit{Core RANK: A}.
\\\textit{https://doi.org/10.1109/ICSM.2013.39}

    \item  \label{C22}  G. Bavota, G. Canfora, M. Di Penta, R. Oliveto, \underline{S. Panichella}. \textbf{*An Empirical Investigation on Documentation Usage Patterns in Maintenance Tasks}. In: \emph{Proceedings of the 29th International Conference on Software Maintenance} (ICSM 2013). Eindhoven, Netherlands.  \textit{Core RANK: A}.
    \\\textit{https://doi.org/10.1109/ICSM.2013.32}

    \item  \label{C23}  G. Canfora, M. Di Penta, S. Giannantonio, R. Oliveto,  \underline{S. Panichella}. \textbf{*YODA: Young and newcOmer Developer Assistant}. In: \emph{Proceedings of the 35th International Conference on Software Engineering} (ICSE 2013). San Francisco, CA, USA.  \textit{Core RANK: A*}. \\\textit{https://doi.org/10.1109/ICSE.2013.6606710}

     \item  \label{C24}  G. Canfora, A. De Lucia, M. Di Penta, R. Oliveto, A. Panichella, \underline{S. Panichella}. \textbf{*Multi-Objective Cross-Project Defect Prediction}. In: \emph{Proceedings of the 7th International Conference on Software Testing, Verification and Validation} (ICST 2013). Luxembourg.  \textit{Core RANK: A}.
     \\\textit{https://doi.org/10.1109/ICST.2013.38}

     \item   \label{C25}  G. Canfora, M. Di Penta, R. Oliveto,  \underline{S. Panichella}. \textbf{*Who is going to Mentor Newcomers in Open Source Projects?}. In: \emph{Proceedings of the 29th ACM SIGSOFT International Symposium on Foundations of Software Engineering} (FSE 2012). Cary, North Carolina, USA.  \textit{Core RANK: A*}. \\\textit{http://doi.acm.org/10.1145/2393596.2393647}

     \item  \label{C26} A. De Lucia, M. Di Penta, R. Oliveto, A. Panichella, \underline{S. Panichella}. \textbf{*Using IR Methods for Labeling Source Code Artifacts: Is It Worthwhile?}. In: \emph{Proceedings of the 20th IEEE International Conference on Program Comprehension} (ICPC), 2012. Passau, Germany.  \textit{Core RANK: C}.
     \\\doi{https://doi.org/10.1109/ICPC.2012.6240488}

     \item  \label{C27}  \underline{S. Panichella}, J. Aponte, M. Di Penta, A. Marcus, G. Canfora. \textbf{Mining source code descriptions from developer communications}. In: \emph{Proceedings of the 20th IEEE International Conference on Program Comprehension} (ICPC), 2012. Passau, Germany.  \textit{Core RANK: C}.
     \\\textit{https://doi.org/10.1109/ICPC.2012.6240510}

     \item  \label{C28}  A. De Lucia, M. Di Penta, R. Oliveto, A. Panichella, \underline{S. Panichella}. \textbf{*Improving IR-based Traceability Recovery Using Smoothing Filters}. In: \emph{Proceedings of the 19th IEEE International Conference on Program Comprehension} (ICPC) 2011. Kingston, ON, Canada.  \textit{Core RANK: C}. \\\textit{https://doi.org/10.1109/ICPC.2011.34}

\end{bibenum}

\textbf{\\\underline{Conference Publications during the bachelor and master studies:}}\\
\begin{bibenum}
     \item  \label{C29}  G. Capobianco, A. De Lucia, R. Oliveto, A. Panichella, \underline{S. Panichella}. \textbf{*On the role of the nouns in IR-based traceability recovery}. In: \emph{Proceedings of the 19th IEEE International Conference on Program Comprehension} (ICPC) 2009. Vancouver, British Columbia, Canada.  \textit{Core RANK: C}.
\\\textit{https://doi.org/10.1109/ICPC.2009.5090038}

     \item  \label{C30}  G. Capobianco, A. De Lucia, R. Oliveto, A. Panichella, \underline{S. Panichella}. \textbf{*Traceability Recovery Using Numerical Analysis}. In: \emph{Proceedings of the 16th IEEE Working Conference on Reverse Engineering} (WCRE) 2009. Lille, France.  \textit{Core RANK: B}.\\ \textit{https://doi.org/10.1109/WCRE.2009.14}
     
\end{bibenum}


%\blankline

%\section{Submitted Papers}
%In papers marked with (*)  the authors are listed in alphabetic order. When such a rule is not followed, authors are listed by contribution.
%\begin{bibenum}
%      \item A. Di Sorbo,  \underline{S. Panichella}, C. V. Alexandru, J. Shimagaki, C. A. Visaggio, G. Canfora, H. Gall.
%         \textbf{What Would Users Change in My App? Summarizing App
%Reviews for Recommending Software Changes}.  In: \emph{24th ACM SIGSOFT International Symposium on the Foundations of Software Engineering}  will be held in Seattle, WA, USA
%\end{bibenum}

%\section{Submitted Papers}
%\begin{bibenum}
    %     \item \underline{S. Panichella}, H. Gall.       \textbf{Summarization Techniques for Code, Changes, and Testing}.  Technical Briefing proposal at  the  \emph{39th International Conference on Software Engineering} (ICSE 2017).
      
%                              \item C. Vassallo, F. Zampetti, D. Romano,  \underline{S. Panichella}, M. Di Penta and A. Zaidman  \textbf{What Types of Build Failures Stop Continuous Delivery? An Empirical Study at ING NL}.    In: \emph{39th International Conference on Software Engineering} (ICSE 2017).
 

%\end{bibenum} 

\blankline

\section{Book chapters:}
\begin{enumerate}
\item Harald C. Gall, Carol V. Alexandru, Adelina Ciurumelea, Giovanni Grano, Christoph Laaber, Sebastiano Panichella, Sebastian Proksch, Gerald Schermann, Carmine Vassallo, Jitong Zhao: \textbf{Data-Driven Decisions and Actions in Today's Software Development}.  	\emph{The Essence of Software Engineering 2018}: 137-168
\end{enumerate}

\blankline

\section{Awards}

Awards as Reviewer:
\begin{enumerate}
\item \textbf{Distinguished Reviewer Award SANER 2018}
\item \textbf{Distinguished Reviewer Award SATToSE 2017}
\end{enumerate}

Best Paper Awards\footnote{
 In papers marked with (*)  the authors are listed in alphabetic order}: %When such a rule is not followed, authors are listed by contribution.
\begin{enumerate}
 \item  Christian Birchler, Nicolas Ganz, Sajad Khatiri, Alessio Gambi and \underline{S. Panichella}. \textbf{Cost-effective Simulation-based Test Selection in Self-driving Cars Software with SDC-Scissor}.  \emph{International Conference on Software Analysis, Evolution, and Reengineering}  (2022) 
 
 \item  G. Grano, T. Titov, \underline{S. Panichella}, H. Gall. \textbf{How High Will It Be? Using Machine Learning Models to Predict Branch Coverage in Automated Testing}.  \emph{MaLTeSQuE}  (collocated with SANER 2018) 
 
\item \textbf{Best paper award}\\
A. De Lucia, M. Di Penta, R. Oliveto, A. Panichella, \underline{S. Panichella}. \textbf{*Improving IR-based Traceability Recovery Using Smoothing Filters}. In: \emph{Proceedings of the 19th IEEE International Conference on Program Comprehension} (ICPC) 2011. Kingston, ON, Canada.  \textit{Core RANK: B}.
\item \textbf{Best tool award}\\
 C. Vassallo, \underline{S. Panichella}, G. Canfora, M. Di Penta. \textbf{CODES: mining sourCe cOde Descriptions from developErs diScussions}. In: \emph{Proceedings of the 36th International Conference on Program Comprehension} (ICPC 2014). Hyderabad, India.  \textit{Core RANK: B}.
 \item \textbf{Best tool award}\\
 L. Pelloni, G. Grano, A. Ciurumelea, \underline{S. Panichella}, F. Palomba, H. Gall. \textbf{BECLoMA: Augmenting Stack Traces with User Review Information.}.  \emph{Proceedings of the  {IEEE} 25th International Conference on Software Analysis, Evolution and Reengineering}  (SANER 2018)
 \item  \textbf{Best tool award}\\Rafael Kallis, Andrea Di Sorbo, Gerardo Canfora and \underline{S. Panichella}. \textbf{Ticket Tagger: Machine Learning Driven Issue Classification}.  \emph{35th IEEE International Conference on Software Maintenance and Evolution (ICSME 2019)}  (Invited to Journal Extension) 
\end{enumerate}

\section{Nominated as Best Paper}
In papers marked with (*)  the authors are listed in alphabetic order. %When such a rule is not followed, authors are listed by contribution.
\begin{enumerate}

\item \label{Cm5} Annibale Panichella, \underline{Sebastiano Panichella}, Gordon Fraser, Anand Ashok Sawant and Vincent Hellendoorn:  \textbf{Revisiting Test Smells in Automatically Generated Tests: Limitations, Pitfalls, and Opportunities}  \emph{International Conference on Software Maintenance and Evolution (ICSME 2020)} 

 \item \label{Cm4}  Muhammad Ilyas Azeem, \underline{S. Panichella}, Andrea Di Sorbo, Alexander Serebrenik, and Qing Wang.  \textbf{ Action-based Recommendation in Pull-request Development}.  \emph{International Conference on Software and System Processes (ICSSP2020)}. Invited for journal extension.

 \item \label{Cm3}  G. Grano, A. Ciurumelea, \underline{S. Panichella}, F. Palomba, H. Gall. \textbf{ Exploring the Integration of User Feedback in Automated Testing of Android Applications.}.  \emph{Proceedings of the  {IEEE} 25th International Conference on Software Analysis, Evolution
               and Reengineering}  (SANER 2018) 
               

      \item \label{Cm2}  C. Vassallo, \underline{S. Panichella}, F. Palomba, S. Proksch, A. Zaidman and H. Gall. \textbf{Context is King: The Developer Perspective on the Usage of Static Analysis Tools.}.  \emph{Proceedings of the  {IEEE} 25th International Conference on Software Analysis, Evolution
               and Reengineering}  (SANER 2018) 
               
                \item     C. Alexandru,  \underline{S. Panichella}, Harald Gall. \textbf{Reducing Redundancies in Multi-Revision Code Analysis}. In: \emph{24th IEEE International Conference on Software Analysis, Evolution, and Reengineering  (SANER) 2017. Klagenfurt, Austria}.  \textit{Core RANK: B}.
     
     \item  \underline{S. Panichella}, G. Bavota, M. Di Penta, G. Canfora, G. Antoniol. \textbf{How Developers' Collaborations Identified from Different Sources Tell us About Code Changes}. In: \emph{Proceedings of the 30th International Conference on Software Maintenance and Evolution} (ICSME 2014). Victoria, Canada.  \textit{Core RANK: A}.
     
    \item  \underline{S. Panichella}, G. Canfora, M. Di Penta, R. Oliveto. \textbf{How the Evolution of Emerging Collaborations Relates to Code Changes: an Empirical Study}. In: \emph{Proceedings of the 36th International Conference on Program Comprehension} (ICPC 2014). Hyderabad, India.  \textit{Core RANK: C}.

    \item  G. Bavota, G. Canfora, M. Di Penta, R. Oliveto, \underline{S. Panichella}. \textbf{*The Evolution of Project Inter-Dependencies in a Software Ecosystem: the Case of Apache}. In: \emph{Proceedings of the 29th International Conference on Software Maintenance} (ICSM 2013). Eindhoven, Netherlands.  \textit{Core RANK: A}.
    
     \item G. Canfora, A. De Lucia, M. Di Penta, R. Oliveto, A. Panichella, \underline{S. Panichella}. \textbf{*Multi-Objective Cross-Project Defect Prediction}. In: \emph{Proceedings of the 7th International Conference on Software Testing, Verification and Validation} (ICST 2013). Luxembourg.  \textit{Core RANK: C}.

     \item A. De Lucia, M. Di Penta, R. Oliveto, A. Panichella, \underline{S. Panichella}. \textbf{*Using IR Methods for Labeling Source Code Artifacts: Is It Worthwhile?}. In: \emph{Proceedings of the 20th IEEE International Conference on Program Comprehension} (ICPC), 2012. Passau, Germany.  \textit{Core RANK: C}.

     \item  A. De Lucia, M. Di Penta, R. Oliveto, A. Panichella, \underline{S. Panichella}. \textbf{*Improving IR-based Traceability Recovery Using Smoothing Filters}. In: \emph{Proceedings of the 19th IEEE International Conference on Program Comprehension} (ICPC) 2011. Kingston, ON, Canada.  \textit{Core RANK: C}.
\end{enumerate}

\blankline




\section{Professional Services and Experiences}

\textbf{Technical Coordinator of EU and National grants:}
\begin{innerlist}
\item Technical coordinator of the H2020 project "COSMOS: DevOps for Complex Cyber-physical Systems"   (recently selected for funding) 
\item 
Technical coordinator of the Innosuisse project "ARIES: Exploiting User Journeys and Testing Automation for Supporting Efficient Energy Service Platforms"   (recently selected for funding) \\
	\end{innerlist}
	

\textbf{Reviewer/opponent of Ph.D. Dissertations:}
\begin{innerlist}
\item Reviewer/opponent of a Ph.D. Dissertation  at University of Tartu, Institute of Computer Science (2019/2020) \\
	\end{innerlist}

\textbf{Keynote Speaker of International Conferences and co-located events:}
\begin{innerlist}
\item Speaker at the Workshop on Dependable DevOps co-located with the SafeComp conference, 2021.\\
\item Keynote speaker at VST 2018 (co-located to SANER 2018) \\(http://vst2018.scch.at/\#program) 
	\\
\end{innerlist}

\textbf{Editor or Co-editor of special Issues at International Journals:}
\begin{innerlist}
\item Editor of Software Track special Issue at Journal of Science of Computer Programming on NLP-based software engineering. 2022
\item Editor of a the special Issue at EMSE entitled 'Software Engineering for Mobile Applications', July 2018.
\item Editor of a the special Issue at IST entitled 'User Feedback and Software Quality in the Mobile Domain',  June 2018. Link to the guest editorial:\\ https://doi.org/10.1016/j.infsof.2019.05.005
	\\
\end{innerlist}

\textbf{Organising Summer School:}
\begin{innerlist}
\item \textbf{1st Summer School on Software Evolution: From Monolithic to Cloud-Native"}. Program available at https://research.tuni.fi/clowee/news/inforte-cloud/ 
	\\
\end{innerlist}

%\textbf{Organising research workshops:}
%\begin{innerlist}
%\end{innerlist}


\textbf{Chair of International Workshop or Tool competitions:}
\begin{innerlist}
\item Chair of the \textit{Workshop on Natural Language-Based Software Engineering Workshop} (NLBSE) - Collocated with ICSE 2022
\item Chair of  the \textit{Workshop on Search-Based Software Testing} (SBST) - Collocated with ICSE 2022
\item Chair of the \textit{Workshop on DevOps Testing for Cyber-Physical Systems - Collocated with ICST 2021 \\(https://devops4cps-testing.github.io/)} 
\item \href{}
{Chair of the Tool Competition at the 
International Workshop on Search-Based Software Testing (SBST 2020 and 2021)} 
       \item \emph{\href{http://cnax.servicelaboratory.ch/}
                   {\textit{Chair of the first International Workshop on Cloud-Native Applications Design and Experience - CNAX 2018
Co-located with UCC 2018 and BDCAT 2018 conferences -}}, Zurich, Switzerland.}

       \item \emph{\href{}
                   {\textit{Chair of the Second International Workshop on Cloud-Native Applications Design and Experience - CNAX 2019
Co-located with UCC 2019 and BDCAT 2019 conferences}}.}
\item \textit{\href{http://www.choose.s-i.ch/events/forum2017/index.html}{Co-organizer of the CHOOSE-forum 2017}} 
\end{innerlist}

%\on{Get rid of all the emphasis (no italics needed).}\\
\textbf{Organising committee member of International Conferences and Workshops:}
\begin{innerlist}
\item Program Committee member of the International Conference on Software Engineering - (ICSE 2023, 2022, 2018)
\item Program Committee member of the  ACM Joint European Software Engineering Conference and Symposium on the Foundations of Software Engineering (ESEC/FSE 2021)
\item Reviewer of Research Track, Industrial Track and Expert Review Panel Member of the IEEE/ACM International Conference on Automated Software Engineering (ASE 2022, 2021, 2017)
\item Program Committee member of  the IEEE Conference on Software Testing, Validation and Verification (ICST 2022, 2020) 
\item Program Committee member of  International Conference on Software Maintenance and Evolution (ICSME 2022, 2018, 2017). 
\item Program Committee member of the  International Conference on Software and Data Technologies (2021)
\item Program Committee member of the International Conference on Program Comprehension (ICPC 2022, ICPC 2020, 2017, 2016, 2015, 2014).
\item Program Committee member of  the International Conference on Mining Software Repositories (MSR 2022, 2020, 2019, 2018, 2016)  
\item Program Committee member of  the Internation Conferance on Software Analysis, Evolution and Reengineering (SANER 2021, 2020, 2019, 2017)
\item Program Committee member of  the  International Workshop on Search-Based Software Testing (SBST 2020, 2019, 2018)   
\item Program Committee member of 1st International Workshop on Machine Learning and Software Engineering in Symbiosis.
\item Program Committee member of  ESEC/FSE 2018 - Formal Demonstration Track.
\item Program Committee member of SBST 2018 (11th International Workshop on Search-Based Software Testing), Gothenburg, Sweden.
\item Program Committee member of the Euromicro Conference on Software Engineering and Advanced Applications (SEAA 2017, 2016, 2015).
\item Program Committee member of the 10th Seminar on Advanced Techniques \& Tools for Software Evolution" (SATToSE 2017), Madrid, Spain.\\
\item Program Committee member of the  Symposium on Search-Based Software Engineering (SSBSE 2021, 2020) 
\item Program Committee member of the  WAISE 2020 (Third International Workshop on Artificial Intelligence Safety Engineering)
\item Program Committee member of  the  of 3rd International Workshop on App Market Analytics (WAMA 2019)  
\item Program Committee member of the  International Workshop on Machine Learning Techniques for Software Quality Evolution (2020)
\item Program Committee member of  the  International Workshop on Robotics Software Engineering (RoSE) 
\item Program Committee member of  the International Conference on the
Quality of Information and Communications Technology 

\end{innerlist}

%\textbf{Session Chair of International Conferences:}
%\begin{innerlist}
%
%       \item \emph{\href{http://saner.aau.at/}
%                   {\textit{of the 24th IEEE International Conference on Software Analysis, Evolution, and Reengineering (SANER 2017 - ERA Track)}}, Klagenfurt, Austria.}
%              \item \emph{\href{}
%                   {\textit{ at the MSR 2018 - technical session, entitled "APIs and Code"}}, Gothenburg, Sweden.}
%
%\end{innerlist}

\textbf{Member of associations:}
\begin{innerlist}
\item Member of the EU Sparc Robotics group - https://sparc-robotics-portal.eu\\
\end{innerlist}

\textbf{Web Chair}
\begin{innerlist}
   \item \emph{21st International Conference on Program Comprehension (ICPC 2013)}, San Francisco, California, USA.\\
\end{innerlist}

\textbf{Editorial Board Member of International Journals:}
\begin{innerlist}
\item \emph{\href{http://onlinelibrary.wiley.com/journal/10.1002/(ISSN)2047-7481}{Journal of Software: evolution and process}}
\end{innerlist}

\textbf{Review Board Member of International Journals:}
\begin{innerlist}
   \item \emph{
              \href{}
                   {Empirical Software Engineering (EMSE)}}.
   \item \emph{
              \href{}
                   {ACM TOSEM Board of Distinguished Reviewers}}.\\
\end{innerlist}




\textbf{Reviewer for the following International Journals:}
\begin{innerlist}
\item \emph{Empirical Software Engineering}.
\item \emph{Transactions on Software Engineering}.
\item \emph{ Transactions on Software Engineering and Methodology}.
\item \emph{Journal of Systems and Software}.
\item \emph{Information and Software Technology}.
\item \emph{Journal of Software: Evolution and Process}.
\item \emph{Science of Computer Programming}.
\item \emph{Journal of Computer Science and Technology}.
\item \emph{Communications of the ACM}
\item \emph{Software Testing, Verification and Reliability}
\item \emph{Transactions on Mobile Computing }
\item \emph{Journal of Object Technology}\\
\end{innerlist} 

%\textbf{Additional reviewer of International Conferences:}
%\begin{innerlist}
%\item 31st IEEE/ACM International Conference on Automated Software Engineering (ASE 2016), Singapore, Singapore.
%\item 30th IEEE/ACM International Conference on Automated Software Engineering (ASE 2015), Lincoln, Nebraska, USA.
%\item 22nd IEEE International Conference on Software Analysis, Evolution, and Reengineering (SANER 2015), Montreal, Canada.\\
%\end{innerlist}


\textbf{Internships}
\begin{innerlist}
   \item \emph{
              \href{}
                   {From 27 May 2013 to 27 July 2013 he has been a visiting researcher at the Ecole Polytechnique de Montr\`{e}al, Canada. Supervisor: Prof. Giuliano Antoniol}}
\end{innerlist}


\textbf{External Reviewer of Grant Applications}
\begin{innerlist}
   \item \emph{
              \href{http://www.frqnt.gouv.qc.ca/accueil}
                   {External Reviewer of projects submitted in the Quebec-Flanders bilateral research cooperation program}}
\end{innerlist}


\textbf{Research Meetings}
\begin{innerlist}
\item \emph{Sebastiano Panichella was invited by the National Institute of Informatics (NII), Japan, to participate in NII Shonan Meeting entitled ``Mobile App Store Analytics" (Japan)}
  \item  \emph{Sebastiano Panichella was invited by the \href{http://www.adesso.de/de/}{Adesso company}, Switzerland, to participate in \textit{``Adesso Quartalsmeeting" 24th feb 2016} (Zurich).}

\end{innerlist}


\section{Grants and EU projects}
%\on{Why are you using href for the entire text?}\\

%\on{Why are you using href for the entire text?}\\
\textbf{EU projects}
\begin{innerlist}
\item Sebastiano Panichella is the technical coordinator for the EU H2020-ICT-2018-20 call, entitled COSMOS, contract no. 957254. \textbf{Description}: Much of the increasing complexity of ICT systems is being driven by the more distributed and heterogeneous nature of these systems, with Cyber-Physical Systems accounting for an increasing portion of Software Ecosystems. This basic premise underpins the COSMOS proposal which focuses on blending best practices DevOps solutions with the development processes used in the CPS context: this will enable the CPS world to deliver software more rapidly and result in more secure and trustworthy systems. 
 %COSMOS brings together a balanced consortium of big industry, SMEs and academics which will develop enhanced DevOps pipelines which target development of CPS software.
The COSMOS CPS pipelines will be validated against 5 use cases provided by industrial partners representing healthcare, avionics, automotive, utility and railway sectors.  
\textbf{Total H2020 project 5MIL EUR, Sebastiano Panichella got direct funding for 770,000 EUR} 
   \item Sebastiano Panichella was 
   %\ins{was}
    partially funded with Gabriele Bavota, Gerardo Canfora, Massimiliano Di Penta, in the \href{��http://www.markosproject.eu/��}
                   {EU FP7-ICT-2011-8 project Markos}, contract no. 317743. Specifically, the MARKOS project aimed to realize the prototype of a service and an interactive application providing an integrated view on the Open Source projects available the on web, focusing on functional, structural and licenses aspects of software code. 
\end{innerlist}
%\newpage
\textbf{Innosuisse projects}
\begin{innerlist}
   \item Sebastiano Panichella is the main research responsible of Innosuisse project ARIES: Exploiting User Journeys and Testing Automation for Supporting Efficient Energy Service Platforms (project Nr. 45548.1 IP-ICT).
ARIES brings together a consortium of two partners: the start-up BOND (\href{https://bond.info/en/}{https://bond.info/en/}) and the ZHAW.
ARIES project will deliver a user-oriented self-adaptive software platform that implements requirements and testing engineering mechanisms to enhance customer experience. ARIES project will be realized in the context of BOND, a Swiss e-bike sharing start-up.\\
\textbf{Total project funding:} Sebastiano Panichella got direct funding for around \textbf{500,000 CHF}\\
%\textbf{Ack:} We personally thank the team of BOND for the very productive and constant research meetings. 
\end{innerlist}

\textbf{SNF projects}
\begin{innerlist}
   \item Sebastiano Panichella obtained funding  (as co-applicant) for
   %\chg{funded}{obtained funding for} 
   the SURF-MobileAppsData SNF (No. 200021$-$166275) project. The goal of the SURF-MobileAppsData project is mining mobile apps data available in app stores to support software engineers in better supporting maintenance and evolution activities for these apps (\textbf{Total SNSF (CHF) 349,926}). \\See page: \href{http://www.ifi.uzh.ch/en/seal/people/panichella/SNF-Projects.html}{http://www.ifi.uzh.ch/en/seal/people/panichella/SNF-Projects.html}
\end{innerlist}

\vspace{5mm}


\section{Talks Given}

\textbf{Talks Given is detailed at https://spanichella.github.io/\#services}

% 
%\begin{bibsection}
%\item \textbf{International Summer School on Software Engineering 2011}\\
%How identify Mentors in software projects? Discussion and perspectives \emph{July 2011}.
%
%\item \textbf{FSE 2012}\\
%Who is going to Mentor Newcomers in Open Source Projects?, \emph{November 2012}.
%
%\item \textbf{ICPC 2012}\\
%Mining source code descriptions from developer communications, \emph{June 2012}.
%
%\item \textbf{ICSE 2013}\\
%YODA: Young and newcOmer Developer Assistant, \emph{May 2013}.
% 
%\item \textbf{ICSM 2013}\\
%Empirical Investigation on Documentation Usage Patterns in Maintenance Tasks, \emph{September}.
%
%\item \textbf{CSER 2013 - Concordia University downtown Montréal (http://concordia.ca)}\\
%Supporting Developers, Mining of Software Repositories, \emph{June}.
%
%\item \textbf{ICPC 2014}\\
%How the Evolution of Emerging Collaborations Relates to Code Changes: an Empirical Study, \emph{June}.
%
%\item \textbf{ICPC 2014}\\
%CODES: mining sourCe cOde Descriptions from developErs diScussions, \emph{June}.
%
%\item \textbf{ICMSE 2014}\\
%How Developers' Collaborations Identified from Different Sources Tell us About Code Changes, \emph{September}.
%
%\item \textbf{ASE 2014}\\
%Recommending Refactorings based on Team Co-Maintenance Patterns, \emph{September}.
%
%\item \textbf{SANER 2015}\\
%Would Static Analysis Tools Help Developers with Code Reviews? \emph{March}.
%
%\item \textbf{ICSME 2015}\\
%How Can I Improve My App? Classifying User Reviews for Software Maintenance and Evolution, \emph{October}.
%
%\item \textbf{ICSME 2015}\\
%Supporting Newcomers in Software Development Projects,  \emph{October}.
%
%\item \textbf{ASE 2015}\\
%Development Emails Content Analyzer: Intention Mining in Developer Discussions,  \emph{November}.
%
%\item \textbf{\href{http://www.lifl.fr/eosese/}{EOSESE 2015}}\\
%Textual Analysis or Natural Language Parsing? A Software Engineering Perspective,  \emph{December}.
%
%\item \textbf{``Adesso Quartalsmeeting" - 2016}\\
%Summarization Techniques for Code, Changes, and Testing,  \emph{February}.
%
%\item \textbf{Invited by Gran Sasso Science Institute, Center of Advanced Studies - 2016}\\
%Systematic Mining of Software Repositories,  \emph{July}.
%   
%\item \textbf{ICSE 2016}\\
%The Impact of Test Case Summaries on Bug Fixing Performance: An Empirical Investigation,  \emph{May}.
%
%\item \textbf{FSE 2016}\\
%ARdoc: App Reviews Development Oriented Classifier,  \emph{November}.
%
%\item \textbf{FSE 2016}\\
%What Would Users Change in My App? Summarizing App Reviews for Recommending Software Changes,  \emph{November}.
%
%\item \textbf{ICSE 2017}\\
%SURF: Summarizer of User Reviews Feedback.,  \emph{May}.
%
%\item \textbf{ICSE 2017}\\
%Analyzing APIs Documentation and Code to Detect Directive Defects,  \emph{May}.
%\item \textbf{VSS 2017}\\
% Summarization Techniques for Code, Change, Testing and User Feedback \emph{December}.
%\item \textbf{VST (collocated with SANER 2018)}\\
% Summarization Techniques for Code, Change, Testing and User Feedback. \emph{March}.
%
%\item \textbf{SBST 2019 (collocated with ICSE 2019)}\\
%DRONE: A Tool to Detect and Repair Directive Defects in Java APIs Documentation.
%\emph{May}.
%
%\item \textbf{ICSE 2019}\\
%The Cloudification Perspectives of Search-based Software Testing
%\emph{May}.
%
%\item \textbf{IC2E 2019}\\
%Quality and Feedback Techniques in Kubernetes Application Engineering
%\emph{June}.
%
%\item \textbf{Did a talk at Cisco Systems GmbH 2019 - https://www.meetup.com/it-IT/Microservices-Zurich/events/262000623/}\\
%Cloud-based testing.
%\emph{July}.
%
%\item \textbf{RE 2020}\\
%Requirements-Collector: Automating Requirements Specification from Elicitation Sessions and User Feedback 
%\emph{September}.
%
%\item \textbf{ICSE 2020}\\
%Java Unit Testing Tool Competition-Eighth Round. IEEE/ACM 42nd International Conference on Software Engineering Workshops (ICSE 2020).
%\emph{May}.
%
%\end{bibsection}

\blankline

\section{PhD Students and Assistants Supervised}

%\on{It would be better to group the PhD, MSc and Bachelors students}\\

\begin{bibsection}

\item \textbf{Sajad Khatiri, PhD student at Zurich University of Applied Science} and USI (Co-advised with Prof. Tonella), Switzerland (from 2021).\\
\textit{- Cost-effective Simulation-based Test Selection inSelf-driving Cars Software with SDC-Scissor.  SANER 2022}.   

\item \textbf{Pooja Rani, PhD student at University of Bern, Switzeerland. }\\
       \textit{- What do class comments tell us? An investigation of comment evolution and practices in Pharo Smalltalk}. Empirical Software Engineering (EMSE 2021).\\
       \textit{- How to Identify Class Comment Types? A Multi-language Approach for Class Comment Classification}. Journal of Systems and Software, 2021. \\
       \textit{- Makar: A Framework for Multi-source Studies based on Unstructured Data}  (SANER 2021).\\
       \textit{- What do Developers Discuss about Code Comments?} International Working Conference on Source Code Analysis and Manipulation 2021 (SCAM).\\

\item \textbf{Christian Birchler, Research assistant at Zurich University of Applied Science}, Switzerland (from 2021). 
\\\textit{- Cost-effective Simulation-based Test Selection inSelf-driving Cars Software with SDC-Scissor.  SANER 2022}. 

\item \textbf{Gabriela Lopez, Research assistant at Zurich University of Applied Science}, Switzerland (from 2021-06). \\
- Working on the Innosuisse ARIES project (Exploiting User Journeys and Testing Automation for Supporting Efficient Energy Service Platforms).\\ 

\item \textbf{Susovita Soumya, Research assistant at Zurich University of Applied Science}, Switzerland (from 2021-02 to 2021-04). 
- Working on the Innosuisse ARIES project (Exploiting User Journeys and Testing Automation for Supporting Efficient Energy Service Platforms).\\ 

\item \textbf{Nicolas Ganz, Research assistant at Zurich University of Applied Science}, Switzerland (from 2021). \\
\textit{- Working on the Innosuisse ARIES project (Exploiting User Journeys and Testing Automation for Supporting Efficient Energy Service Platforms)}\\
\textit{- Cost-effective Simulation-based Test Selection inSelf-driving Cars Software with SDC-Scissor.  SANER 2022}. 

\item \textbf{Muhammad Ilyas Azeem, PhD student at Laboratory for Internet Software Technologies}, Institute of Software Chinese Academy of Sciences, Beijing 100190, China. .\\
       \textit{- Action-based Recommendation in Pull-request Development}  (ICSSP 2020).\\

\item \textbf{Carol V. Alexandru, PhD student at University of Zurich, Switzerland, 2016}.\\
       \textit{- A Search-based Training Algorithm for Cost-aware Defect Prediction} (GECCO 2016).\\
              \textit{- What Would Users Change in My App? Summarizing App Reviews for Recommending Software Changes} (FSE 2016).\\
               \textit{- ARdoc: App Reviews Development Oriented Classifier} (FSE 2016).\\
        \textit{- Exploring Deep Learning Techniques for Supporting the Mining of information
        in Structured and Unstructured Data}.\\
        \textit{- Reducing Redundancies in Multi-Revision Code Analysis (SANER 2017)}. \\
        \textit{- Replicating Parser Behavior using Neural Machine Translation} (ICPC 2017).\\
        \textit{- Redundancy-free Analysis of Multi-revision Software Artifacts}.  EMSE 2018\\
        \textit{- On the Usage of Pythonic Idioms. Artifacts}.  ONWARD 2018\\
\item \textbf{Giovanni Grano, PhD student at University of Zurich, Switzerland, 2017}.\\ 
          \textit{- Investigating the Criticality of User Reported Issues through their Relations with App Rating. Journal of Software: Evolution and Process (JSEP)}  
                 \\   \textit{- Testing with Fewer Resources: An Adaptive Approach to Performance-Aware Test Case Generation.  Transactions on Software Engineering (TSE)} 
         \\  \textit{- Branch Coverage Prediction in Automated Testing.  Journal of Software: Evolution and Process (JSEP).} \\
         \textit{- Exploring the Integration of User Feedback in Automated Testing of Android Applications}  (SANER 2018).  \\  
         \textit{- BECLoMA: Augmenting Stack Traces with User Review Information}  (SANER 2018).  \\
         \textit{- How High Will It Be? Using Machine Learning Models to Predict Branch Coverage in Automated Testing.  \emph{MaLTeSQuE}}  (collocated with SANER 2018).  \\
       \textit{- Android Apps and User Feedback: a Dataset for Software Evolution and Quality Improvement}  (WAMA 2017). \\    
         
         
\item \textbf{Adelina Ciurumelea, PhD student at University of Zurich, Switzerland, 2016}.\\
          \textit{- Exploring the Integration of User Feedback in Automated Testing of Android Applications}  (SANER 2018).\\    
         \textit{- BECLoMA: Augmenting Stack Traces with User Review Information}  (SANER 2018).  \\
              \textit{- Recommending and Localizing Code Changes for Mobile Apps based on User Reviews (ICSE 2017).}\\   
       \textit{- Analyzing Reviews and Code of Mobile Apps for better Release Planning} (SANER 2017).
  

\item \textbf{Carmine Vassallo, PhD student at University of Zurich, Switzerland, 2016}.\\
        \textit{- How Developers Engage with Static Analysis Tools in Different Contexts .  Empirical Software Engineering Journal.}\\ 
       \textit{- A Tale of CI Build Failures: an Open Source and a Financial Organization Perspective} (ICSME 2017). \\
       \textit{- Context is King: The Developer Perspective on the Usage of Static Analysis Tools} (SANER 2018). \\
       \textit{- How Developers Engage with Static Analysis Tools in Different Contexts}. \emph{Empirical Software Engineering (EMSE)} 2019
   
\item \textbf{Gerald Schermann, PhD student at University of Zurich, Switzerland, 2015}.\\
        \textit{Discovering Loners and Phantoms in Commit and Issue Data} (ICPC 2015).


\item \textbf{Andrea Di Sorbo, PhD student at University of Sannio, Italy, 2016}.\\
        \textit{- How Can I Improve My App? Classifying User Reviews for Software Maintenance and Evolution} (ICSME 2015).  \\
                \textit{- Development Emails Content Analyzer: Intention Mining in Developer Discussions} (ASE 2015).   \\
                 \textit{- DECA: Development Emails Content Analyzer} (ICSE 2016).  \\
                               \textit{- What Would Users Change in My App? Summarizing App Reviews for Recommending Software Changes} (FSE 2016).\\
                               \textit{- ARdoc: App Reviews Development Oriented Classifier} (FSE 2016).\\
                               \textit{- SURF: Summarizer of User Reviews Feedback} (ICSE 2017).
                                      \textit{- Android Apps and User Feedback: a Dataset for Software Evolution and Quality Improvement}  (WAMA 2017). \\    

\section{Master Students Supervised}

\item  \textbf{Gabriela Lopez, Master student at University of Zurich, Italy. }\\
- Automated change analysis. Zurich, Switzerland. 2021. \\

\item  \textbf{Xiao'ao Song, Master student at at University of Zurich, Switzerland, 2021}. \\
\item \textbf{Neeraj Kumar, Master student at at University of Zurich, Switzerland, 2021}. \\

\item \textbf{Bill Bosshard, Master student at University of Zurich, Switzerland, 2019}.\\

\item \textbf{Atif Ghulam, Master student at University of Zurich, Switzerland, 2019}.\\

\item \textbf{Rafael Kallis, Master student at University of Zurich, Switzerland, 2019}.\\
                 \textit{- Ticket Tagger: Machine Learning Driven Issue Classification}  (ICSME 2019).  \\
                 \textit{- Predicting Issue Types on GitHub.}  (SCP 2021).  \\

 \item \textbf{Timofey Titov, Master student at University of Zurich, Switzerland, 2017}.\\
                 \textit{- Branch Coverage Prediction in Automated Testing}  (JSEP 2018).  \\
                 \textit{- BECLoMA: Augmenting Stack Traces with User Review Information}  (SANER 2018).  \\
        \item \textbf{Alessandro Rigamonti, Master student at University of Zurich, Switzerland, 2017}.\\
        \textit{Develop search-based approaches to better predict change and defect prone classes}. Zurich, Switzerland. 2016.

\item \textbf{Carmine Vassallo, Master student at University of Sannio, Italy}.\\
        \textit{CODES: mining source code descriptions from developers discussions. (ICPC 2014)}

\item \textbf{Te Tan, master student at University of Zurich, Switzerland, 2017}.\\
        \textit{Advised on a Work on App Store Mining.}. 
                        \item \textbf{Simon Taennler, master student at University of Zurich, Switzerland, 2017}.\\
        \textit{Advised on a Work on App Store Mining.}. 

\section{Bachelor Students Supervised}
\item \textbf{Timothy Zimmermann, bachelor student at University of Zurich, Switzerland}, 2021.
\item \textbf{Tim Moser, bachelor student at University of Zurich, Switzerland}, 2021.
\item \textbf{Farul Acibal, bachelor student at University of Zurich, 2018}.
\item \textbf{Nik Zaugg, bachelor student at University of Zurich, Switzerland, 2018}.
 \\
        \textit{An Empirical Investigation of Relevant Changes and Automation Needs in Modern Code Review. Empirical Software Engineering (EMSE 2020)}. 
\item \textbf{Gulshan Kundra, master student at LUT, Finland, 2018}.
\item \textbf{Ivan Taraca, bachelor student at University of Zurich, Switzerland, 2017}.\\
        \textit{Tool-support for Test Cases Summaries generator and Enhancements}. 
\item \textbf{Alexander Hofmann, bachelor student at University of Zurich, Switzerland, 2017}.\\
        \textit{ChangeAdvisor - 
A tool for Recommending and Localizing
Change Requests for Mobile Apps based on
User Reviews.}. 

                        \item \textbf{Antonio Galluccio, Bachelor student at University of Zurich, Switzerland, 2017}.\\
        \textit{Toward Generating Test Case Summaries.}. 
         
        
                                \item \textbf{Lucas Pelloni, Bachelor student at University of Zurich, Switzerland, 2017}.\\
                 \textit{- BECLoMA: Augmenting Stack Traces with User Review Information}  (SANER 2018).  \\
                       
                \item \textbf{Andreas Schaufelbuhl, Bachelor student at University of Zurich, Switzerland, 2016}.\\
        \textit{Analyzing Reviews and Code of Mobile Apps for better Release Planning}. (SANER 2017).

\item \textbf{Stefano Giannantonio, Bachelor student at University of Molise, Italy}.\\
        \textit{YODA: Young and newcOmer Developer Assistant. (ICSE 2013)}


\end{bibsection}

\blankline

\section{Skills, Competencies gained during the PhD}

\textbf{Main Competencies Gained:}\\\\
\textit{1) Machine Learning, Text Analysis and Natural Language Processing}\\\\
He is an expert in Mining of Software repositories and successfully adopted/conceived  
tools based on Machine Learning (ADTree, Logistic Regression etc.) methods, Natural Language Processing (Stanford NLP parser, Stanford NLP POS Tagger etc.) techniques and Text Analysis (e.g. Vector Space Model, Latent Dirichlet Allocation, Latent Semantic Indexing Jensen and Shannon Model etc.) techniques. 
For example, a specific example of application of such competencies is represented by the implementation of the tool \href{https://spanichella.github.io/tools.html}{ARdoc} (App Reviews Development Oriented Classifier) which is a Java tool that automatically recognizes natural language fragments in user reviews that are relevant for developers to evolve their applications. Specifically, natural language fragments are extracted according to a taxonomy of app reviews categories that are relevant to software maintenance and evolution. The categories were defined in our previous paper entitled \href{https://spanichella.github.io/index.html\#publications}{\textit{``How Can I Improve My App? Classifying User Reviews for Software Maintenance and Evolution''}}  and are: (i) Information Giving, (ii) Information Seeking, (iii) Feature Request and (iv) Problem Discovery. ARdoc implements an approach that merges three techniques: (1) Natural Language Processing, (2) Text Analysis and (3) Sentiment Analysis to automatically classify app reviews into the proposed categories. The purpose of ARdoc is to capture informative user reviews (requesting a new feature, description of a problem, or proposing a solution) and consequently to allow developers to better manage the information contained in user reviews.\\

\textit{2) Genetic Algorithms in SE}\\

His research has yielded approaches to predict future defects
in software artifacts based on historical information, thus
assisting companies in effectively allocating limited development
resources and developers in reviewing each others
code changes. Developers are unlikely to devote the same
effort to inspect each software artifact predicted to contain
defects, since the effort varies with the artifacts %\on{Why is there a question mark here?}
 size (cost)
and the number of defects it exhibits (effectiveness). He
adopted Genetic Algorithms (GAs) for training prediction
models to maximize their cost-effectiveness. The evaluation of the approach was performed on
 on two well-known models, Regression
Tree and Generalized Linear Model, and predict defects between
multiple releases of six open source projects. The achieved
results show that regression models trained by GAs significantly outperform their traditional counterparts, improving the cost-effectiveness by up to 240\%. Often the top 10\% of
predicted lines of code contain up to twice as many defects. \\\\

\textit{3) Social Network Analysis }\\\\

He is also an expert in Social Network Analysis (SNA) and has successfully used such information for profiling developers/expert in developers' SNA.  See  for example the papers  \textit{How the Evolution of Emerging Collaborations Relates to Code Changes: an Empirical Study} and \textit{Who is going to Mentor Newcomers in Open Source Projects?} and download the related tool \href{https://spanichella.github.io/tools.html}{Yoda} (Young and newcOmer Developer Assistant) which is an Eclipse plugin (available in \href{https://spanichella.github.io/tools.html}{https://spanichella.github.io/tools.html}) able to profile expert in developers' SNA.\\

\textbf{4) Statistics:}\\
During the PhD experience,  because of his work in ``Empirical software engineering",
he gained good experience in Statistics (the R environment was the main tool used for such purposes).
He widely used several statistical tests (parametric and non) for formulating hypothesis
and demonstrating the statistical significance (or superiority) of the proposed techniques.\\


\textbf{5) Main Programming Languages:}\\
He currently uses for his work programming languages such as Java, Python, Perl (basic level). He is very skilled in scripting languages like R, Matlab (medium level), Weka, RWeka.\\

\textit{6) Other technologies}\\

Other languages that he used during his academic experience are C, C++, Perl, Scilab, 
Pascal, Visual basic, Prolog, Lisp, PHP,  JSP and Servlet. I also have strong experience with scientific software
and tools, such as Matlab, R, Weka, that are widely used to build 
mathematical models through machine learning techniques (including 
defect prediction models). Other technologies and tools that he used
during the academic years include SVN/GIT and DBMS, PostgreSQL, Gerrit
 code review Tool. \\\\
 
He works currently without problem with different  Operating Systems, like Windows, Mac OS, and Linux (I know very 
well the Ubuntu distribution). \\\\

He is also very familiar with SQL (He  currently use for his research work PostgreSQL).
He proficiently use GIT/SVN as versioning systems. He also wrote a series of research paper using Latex tool as main reference.


\section{Research Tools Implemented}
\textbf{A complete list of implemented Tools and Dataset available at:} \\https://spanichella.github.io/tools.html.\\

\section{Languages}

Sebastiano Panichella currently speak three languages: Italian (mather tongue), English (\textit{C1}) and German (\textit{B1}). He is still studying German for achieving the level B2 during 2021 and C1 during 2022.
\blankline


\section{Teaching Experience}

\vspace{2mm}


\textbf{TEACHING (UZH, ZHAW, and University of Sannio) activities \& Achievements:}\\\\
\textbf{University of Zurich:} \\
-   Lecturer and co-lecturer for the Software Maintenance and Evolution course in 2014, 2015, 2016, 2017, 2018, 2019, 2020, 2021\\   \textit{Learning Goals}: During the course Sebastiano teach to
the students the foundations of software evolution and maintenance, by integrating recent research in both cloud computing and
software engineering fields, thus transferring to students also this recent research outputs (in form of papers, datasets, tools and
prototypes). This includes successful aged (i.e. legacy software) or cloud-based software systems, object-oriented reengineering,
refactoring, change patterns, empirical analysis of software, classification/prediction models, software quality analysis.\\
\textbf{Zurich University of Applied Science (ZHAW):}\\
- Cloud Computing course - CCP2 2020\\
- INF-Prog1 2020. Learning Goals: The main features of the Python program language.\\
- Co-lecturer for the CAS Information Engineering in 2018, 2019, 2020.
Learning Goals: The main features of the Python program language.\\
- Lab Instructor for the Programming course in Java in 2018, 2019, 2020.
Learning Goals: The main features of the Java program language.\\


\href{http://www.unisa.it}{\textbf{University of Sannio}}:\\

- \textit{Lab Instructor} (December 2013) for the Programming Techniques course of Professor Gerardo Canfora\\   \textit{Learning Goals}:   The Languages ​​and Grammars, JavaCC parser.\\
- \textit{Teaching Assistant } for the Software Engineering course of Prof. Massimiliano Di Penta:\\   \textit{Learning Goals}:   
Recovering Traceability Links via Information Retrieval Methods\\



\blankline

\section{Professional Memberships}
    \begin{innerlist}
    \item IEEE Membership (2011--present)
    \item ACM Membership (2019--2020)
    \end{innerlist}

\blankline
 
\section{References} 

\halfblankline 

\textbf{Andy Zaidman, Ph.D}
(e-mail: \href{mailto:A.E.Zaidman@tudelft.nl}{A.E.Zaidman@tudelft.nl}) 
%
\begin{innerlist}
    \item Professor,
        \href{}{University of Delft}

    %\item[$\diamond$] Binzmuhlestrasse 14, CH-8050 Zurich (Switzerland).

    \item[$\star$] \emph{Dr.~Zaidman, co-author of some publications.}
\end{innerlist}

\textbf{Sean Murphy, Ph.D}
(e-mail: \href{mailto:murp@zhaw.ch}{murp@zhaw.ch}) 
%
\begin{innerlist}
    \item Senior Researcher,
        \href{}{Zurich University of Applied Science}

    \item[$\star$] \emph{Dr.~Murphy, close collegue at the Zurich University of Applied Science.}
\end{innerlist}

\href
{http://www.gerardocanfora.net/}
{\textbf{Gerardo Canfora, Ph.D}}
(e-mail: \href{mailto:canfora@unisannio.it}{canfora@unisannio.it})
%
\begin{innerlist}
    \item Full Professor,
        \href{http://www.unisannio.it/}{University of Sannio}

    \item[$\diamond$] Palazzo ex Poste, Via Traiano, I-82100 Benevento (Italy).

    \item[$\star$] \emph{Dr. Canfora}\\%, PhD advisor.}\\
\end{innerlist}

\textbf{Harald C. Gall, Ph.D}
(e-mail: \href{mailto:gall@ifi.uzh.ch}{gall@ifi.uzh.ch}) 
%
\begin{innerlist}
    \item Full Professor,
        \href{http://www.uzh.ch/index_en.html}{University of Zurich}

    \item[$\diamond$] Binzmuhlestrasse 14, CH-8050 Zurich (Switzerland).

    \item[$\star$] \emph{Dr.~Gall, co-author of some publications at the University of Zurich.}
\end{innerlist}

\halfblankline

\href
{http://www.utdallas.edu/~amarcus/}
{\textbf{Andrian Marcus, Ph.D}}
(e-mail:~\href{mailto:amarcus@utdallas.edu}{amarcus@utdallas.edu}; phone:~(972-883-4246)
%
\begin{innerlist}
    \item Associate Professor,
        \href{http://www.utdallas.edu/}{University of Texas at Dallas}

    \item[$\diamond$] 800 W. Campbell Road; MS EC31 Richardson, TX 75080 U.S.A.

    \item[$\star$] \emph{Dr.~Marcus, member of dissertation committee and co-author of some publications.}
\end{innerlist}

\halfblankline

\textbf{Giuliano Antoniol, Ph.D}
(e-mail: \href{mailto:antoniol@ieee.org}{antoniol@ieee.org}) 
%
\begin{innerlist}
    \item Professor,
        \href{}{Ecole Polytechnique de Montreal}

    %\item[$\diamond$] Binzmuhlestrasse 14, CH-8050 Zurich (Switzerland).

    \item[$\star$] \emph{Dr.~Antoniol, co-author of some publications.}
\end{innerlist}


\halfblankline

\href
{http://www.rcost.unisannio.it/mdipenta/}
{\textbf{ (Advisor During the PhD) Massimiliano Di Penta, Ph.D}}
(e-mail: \href{mailto:dipenta@unisannio.it}{dipenta@unisannio.it})
%
\begin{innerlist}
    \item Associate Professor,
        \href{http://www.unisannio.it/}{University of Sannio}

    \item[$\diamond$] Palazzo ex Poste, Via Traiano, I-82100 Benevento (Italy).

    \item[$\star$] \emph{Dr. Di Penta, PhD advisor.}\\
\end{innerlist}
%\halfblankline

%\halfblankline

\href
{http://www.dmi.unisa.it/people/delucia/www/}
{\textbf{Andrea De Lucia, Ph.D}}
(e-mail:~\href{mailto:adelucia@unisa.it}{adelucia@unisa.it}; phone:~+39 089963376)
%
\begin{innerlist}
    \item Professor,
        \href{http://www.unisa.it}{University of Salerno}

    \item[$\diamond$] Via Giovanni Paolo II, 132 - 84084 Fisciano (SA), Italy.

    \item[$\star$] \emph{Prof.~De Lucia, co-author of some publications.}
\end{innerlist}


%%\halfblankline
%
%%\href
%%{http://www.distat.unimol.it/people/oliveto/}
%%{\textbf{Rocco Oliveto, Ph.D}}
%%(e-mail:~\href{mailto:rocco.oliveto@unimol.it}{rocco.oliveto@unimol.it}; phone:~+39 0874 404123)
%%%
%%\begin{innerlist}
%%    \item Assistant Professor,
%%        \href{http://www.unimol.it}{University of Molise}
%%
%%    \item[$\diamond$] C.da Fonte Lappone - 86090 Pesche (IS), Italy.
%%
%%    \item[$\star$] \emph{Dr.~Oliveto, co-author of some publications.}
%%\end{innerlist}


\halfblankline
\blankline


%\hspace{-3cm}\vspace{-0.5cm}\begin{minipage}{1.25\linewidth}
%I authorise the handling of my personal data pursuant to the Personal Data Protection Code – Legislative Decree n. 196/2003.
%
%\blankline
%
%\blankline
%
%February 1st 2013.
%\end{minipage}
%\begin{figure}[h]
%\hspace{-3cm}\includegraphics[width=0.25\linewidth]{sign.png}
%\end{figure}

\end{document}

%%%%%%%%%%%%%%%%%%%%%%%%%% End CV Document %%%%%%%%%%%%%%%%%%%%%%%%%%%%%

%----------------------------------------------------------------------%
% The following is copyright and licensing information for
% redistribution of this LaTeX source code; it also includes a liability
% statement. If this source code is not being redistributed to others,
% it may be omitted. It has no effect on the function of the above code.
%----------------------------------------------------------------------%
% Copyright (c) 2007, 2008, 2009, 2010, 2011 by Theodore P. Pavlic
%
% Unless otherwise expressly stated, this work is licensed under the
% Creative Commons Attribution-Noncommercial 3.0 United States License. To
% view a copy of this license, visit
% http://creativecommons.org/licenses/by-nc/3.0/us/ or send a letter to
% Creative Commons, 171 Second Street, Suite 300, San Francisco,
% California, 94105, USA.
%
% THE SOFTWARE IS PROVIDED "AS IS", WITHOUT WARRANTY OF ANY KIND, EXPRESS
% OR IMPLIED, INCLUDING BUT NOT LIMITED TO THE WARRANTIES OF
% MERCHANTABILITY, FITNESS FOR A PARTICULAR PURPOSE AND NONINFRINGEMENT.
% IN NO EVENT SHALL THE AUTHORS OR COPYRIGHT HOLDERS BE LIABLE FOR ANY
% CLAIM, DAMAGES OR OTHER LIABILITY, WHETHER IN AN ACTION OF CONTRACT,
% TORT OR OTHERWISE, ARISING FROM, OUT OF OR IN CONNECTION WITH THE
% SOFTWARE OR THE USE OR OTHER DEALINGS IN THE SOFTWARE.
%----------------------------------------------------------------------%
